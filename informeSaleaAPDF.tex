\documentclass[]{article}
\usepackage{lmodern}
\usepackage{amssymb,amsmath}
\usepackage{ifxetex,ifluatex}
\usepackage{fixltx2e} % provides \textsubscript
\ifnum 0\ifxetex 1\fi\ifluatex 1\fi=0 % if pdftex
  \usepackage[T1]{fontenc}
  \usepackage[utf8]{inputenc}
\else % if luatex or xelatex
  \ifxetex
    \usepackage{mathspec}
  \else
    \usepackage{fontspec}
  \fi
  \defaultfontfeatures{Ligatures=TeX,Scale=MatchLowercase}
\fi
% use upquote if available, for straight quotes in verbatim environments
\IfFileExists{upquote.sty}{\usepackage{upquote}}{}
% use microtype if available
\IfFileExists{microtype.sty}{%
\usepackage{microtype}
\UseMicrotypeSet[protrusion]{basicmath} % disable protrusion for tt fonts
}{}
\usepackage[margin=1in]{geometry}
\usepackage{hyperref}
\hypersetup{unicode=true,
            pdftitle={TP1, Modelo Lineal},
            pdfauthor={Gonzalo Barrera Borla y Octavio Martín Duarte},
            pdfborder={0 0 0},
            breaklinks=true}
\urlstyle{same}  % don't use monospace font for urls
\usepackage{color}
\usepackage{fancyvrb}
\newcommand{\VerbBar}{|}
\newcommand{\VERB}{\Verb[commandchars=\\\{\}]}
\DefineVerbatimEnvironment{Highlighting}{Verbatim}{commandchars=\\\{\}}
% Add ',fontsize=\small' for more characters per line
\usepackage{framed}
\definecolor{shadecolor}{RGB}{248,248,248}
\newenvironment{Shaded}{\begin{snugshade}}{\end{snugshade}}
\newcommand{\AlertTok}[1]{\textcolor[rgb]{0.94,0.16,0.16}{#1}}
\newcommand{\AnnotationTok}[1]{\textcolor[rgb]{0.56,0.35,0.01}{\textbf{\textit{#1}}}}
\newcommand{\AttributeTok}[1]{\textcolor[rgb]{0.77,0.63,0.00}{#1}}
\newcommand{\BaseNTok}[1]{\textcolor[rgb]{0.00,0.00,0.81}{#1}}
\newcommand{\BuiltInTok}[1]{#1}
\newcommand{\CharTok}[1]{\textcolor[rgb]{0.31,0.60,0.02}{#1}}
\newcommand{\CommentTok}[1]{\textcolor[rgb]{0.56,0.35,0.01}{\textit{#1}}}
\newcommand{\CommentVarTok}[1]{\textcolor[rgb]{0.56,0.35,0.01}{\textbf{\textit{#1}}}}
\newcommand{\ConstantTok}[1]{\textcolor[rgb]{0.00,0.00,0.00}{#1}}
\newcommand{\ControlFlowTok}[1]{\textcolor[rgb]{0.13,0.29,0.53}{\textbf{#1}}}
\newcommand{\DataTypeTok}[1]{\textcolor[rgb]{0.13,0.29,0.53}{#1}}
\newcommand{\DecValTok}[1]{\textcolor[rgb]{0.00,0.00,0.81}{#1}}
\newcommand{\DocumentationTok}[1]{\textcolor[rgb]{0.56,0.35,0.01}{\textbf{\textit{#1}}}}
\newcommand{\ErrorTok}[1]{\textcolor[rgb]{0.64,0.00,0.00}{\textbf{#1}}}
\newcommand{\ExtensionTok}[1]{#1}
\newcommand{\FloatTok}[1]{\textcolor[rgb]{0.00,0.00,0.81}{#1}}
\newcommand{\FunctionTok}[1]{\textcolor[rgb]{0.00,0.00,0.00}{#1}}
\newcommand{\ImportTok}[1]{#1}
\newcommand{\InformationTok}[1]{\textcolor[rgb]{0.56,0.35,0.01}{\textbf{\textit{#1}}}}
\newcommand{\KeywordTok}[1]{\textcolor[rgb]{0.13,0.29,0.53}{\textbf{#1}}}
\newcommand{\NormalTok}[1]{#1}
\newcommand{\OperatorTok}[1]{\textcolor[rgb]{0.81,0.36,0.00}{\textbf{#1}}}
\newcommand{\OtherTok}[1]{\textcolor[rgb]{0.56,0.35,0.01}{#1}}
\newcommand{\PreprocessorTok}[1]{\textcolor[rgb]{0.56,0.35,0.01}{\textit{#1}}}
\newcommand{\RegionMarkerTok}[1]{#1}
\newcommand{\SpecialCharTok}[1]{\textcolor[rgb]{0.00,0.00,0.00}{#1}}
\newcommand{\SpecialStringTok}[1]{\textcolor[rgb]{0.31,0.60,0.02}{#1}}
\newcommand{\StringTok}[1]{\textcolor[rgb]{0.31,0.60,0.02}{#1}}
\newcommand{\VariableTok}[1]{\textcolor[rgb]{0.00,0.00,0.00}{#1}}
\newcommand{\VerbatimStringTok}[1]{\textcolor[rgb]{0.31,0.60,0.02}{#1}}
\newcommand{\WarningTok}[1]{\textcolor[rgb]{0.56,0.35,0.01}{\textbf{\textit{#1}}}}
\usepackage{graphicx,grffile}
\makeatletter
\def\maxwidth{\ifdim\Gin@nat@width>\linewidth\linewidth\else\Gin@nat@width\fi}
\def\maxheight{\ifdim\Gin@nat@height>\textheight\textheight\else\Gin@nat@height\fi}
\makeatother
% Scale images if necessary, so that they will not overflow the page
% margins by default, and it is still possible to overwrite the defaults
% using explicit options in \includegraphics[width, height, ...]{}
\setkeys{Gin}{width=\maxwidth,height=\maxheight,keepaspectratio}
\IfFileExists{parskip.sty}{%
\usepackage{parskip}
}{% else
\setlength{\parindent}{0pt}
\setlength{\parskip}{6pt plus 2pt minus 1pt}
}
\setlength{\emergencystretch}{3em}  % prevent overfull lines
\providecommand{\tightlist}{%
  \setlength{\itemsep}{0pt}\setlength{\parskip}{0pt}}
\setcounter{secnumdepth}{0}
% Redefines (sub)paragraphs to behave more like sections
\ifx\paragraph\undefined\else
\let\oldparagraph\paragraph
\renewcommand{\paragraph}[1]{\oldparagraph{#1}\mbox{}}
\fi
\ifx\subparagraph\undefined\else
\let\oldsubparagraph\subparagraph
\renewcommand{\subparagraph}[1]{\oldsubparagraph{#1}\mbox{}}
\fi

%%% Use protect on footnotes to avoid problems with footnotes in titles
\let\rmarkdownfootnote\footnote%
\def\footnote{\protect\rmarkdownfootnote}

%%% Change title format to be more compact
\usepackage{titling}

% Create subtitle command for use in maketitle
\newcommand{\subtitle}[1]{
  \posttitle{
    \begin{center}\large#1\end{center}
    }
}

\setlength{\droptitle}{-2em}

  \title{TP1, Modelo Lineal}
    \pretitle{\vspace{\droptitle}\centering\huge}
  \posttitle{\par}
    \author{Gonzalo Barrera Borla y Octavio Martín Duarte}
    \preauthor{\centering\large\emph}
  \postauthor{\par}
      \predate{\centering\large\emph}
  \postdate{\par}
    \date{5 de Junio de 2019}


\begin{document}
\maketitle

\hypertarget{consignas}{%
\section{Consignas}\label{consignas}}

Escriba y entregue en un \(script\) un programa de R que haga lo
siguiente.

\hypertarget{a-fije-la-semilla}{%
\subsection{a) Fije la Semilla}\label{a-fije-la-semilla}}

\hypertarget{i.-para-n10-genere-n-datos-y_i-que-sigan-el-modelo-lineal-y_i-42-cdot-x_i1---3-cdot-x_i2-05-cdot-x_i3-varepsilon_i-1-leq-i-leq-n-.-donde}{%
\subsubsection{\texorpdfstring{i. Para \(n=10\) genere \(n\) datos
\(y_i\) que sigan el modelo lineal
\(y_i = 4+2 \cdot x_{i1} - 3 \cdot x_{i2} + 0,5 \cdot x_{i3} + {\varepsilon}_i\),
\(1 \leq i \leq n\) .
Donde}{i. Para n=10 genere n datos y\_i que sigan el modelo lineal y\_i = 4+2 \textbackslash cdot x\_\{i1\} - 3 \textbackslash cdot x\_\{i2\} + 0,5 \textbackslash cdot x\_\{i3\} + \{\textbackslash varepsilon\}\_i, 1 \textbackslash leq i \textbackslash leq n . Donde}}\label{i.-para-n10-genere-n-datos-y_i-que-sigan-el-modelo-lineal-y_i-42-cdot-x_i1---3-cdot-x_i2-05-cdot-x_i3-varepsilon_i-1-leq-i-leq-n-.-donde}}

\begin{itemize}
\tightlist
\item
  \(x_{1i} \sim \mathcal{U}(-5,5) ,iid.\)
\item
  \(x_{2i} \sim \mathcal{U}(-5,5) ,iid.\)
\item
  \(x_{3i} \sim \mathcal{U}(-5,5) ,iid.\)
\item
  \(x_{4i} \sim \mathcal{U}(-5,5) ,iid.\)
\item
  \(\varepsilon_{1i} \sim \mathcal{E}(\lambda=1/2)-2 ,iid.\)
\end{itemize}

\hypertarget{ii.-ajuste-el-modelo-y_i-beta_0beta_1-cdot-x_i1--beta_2-cdot-x_i2-beta_3-cdot-x_i3-beta_4-cdot-x_i4-u_i}{%
\subsubsection{\texorpdfstring{ii. Ajuste el modelo
\(y_i = \beta_0+\beta_1 \cdot x_{i1} -\beta_{2} \cdot x_{i2} + \beta_{3} \cdot x_{i3} + \beta_{4} \cdot x_{i4} + u_i\)}{ii. Ajuste el modelo y\_i = \textbackslash beta\_0+\textbackslash beta\_1 \textbackslash cdot x\_\{i1\} -\textbackslash beta\_\{2\} \textbackslash cdot x\_\{i2\} + \textbackslash beta\_\{3\} \textbackslash cdot x\_\{i3\} + \textbackslash beta\_\{4\} \textbackslash cdot x\_\{i4\} + u\_i}}\label{ii.-ajuste-el-modelo-y_i-beta_0beta_1-cdot-x_i1--beta_2-cdot-x_i2-beta_3-cdot-x_i3-beta_4-cdot-x_i4-u_i}}

\hypertarget{iii.-guarde-los-parametros-estimados.}{%
\subsubsection{iii. Guarde los parámetros
estimados.}\label{iii.-guarde-los-parametros-estimados.}}

\hypertarget{iv.-construya-el-intervalo-de-confianza-de-nivel-0.90-para-el-parametro-beta_1-y-para-el-parametro-beta_4-asumiendo-normalidad-de-los-errores.-contienen-estos-intervalos-a-los-verdaderos-parametros-para-la-muestra-simulada-guarde-en-un-nuevo-objeto-un-uno-si-lo-contiene-y-un-cero-sino-para-cada-uno-de-los-dos-intervalos.}{%
\subsubsection{\texorpdfstring{iv. Construya el intervalo de confianza
de nivel 0.90 para el parámetro \(\beta_1\) y para el parámetro
\(\beta_4\) asumiendo normalidad de los errores. ¿Contienen estos
intervalos a los verdaderos parámetros para la muestra simulada? Guarde
en un nuevo objeto un uno si lo contiene, y un cero sino, para cada uno
de los dos
intervalos.}{iv. Construya el intervalo de confianza de nivel 0.90 para el parámetro \textbackslash beta\_1 y para el parámetro \textbackslash beta\_4 asumiendo normalidad de los errores. ¿Contienen estos intervalos a los verdaderos parámetros para la muestra simulada? Guarde en un nuevo objeto un uno si lo contiene, y un cero sino, para cada uno de los dos intervalos.}}\label{iv.-construya-el-intervalo-de-confianza-de-nivel-0.90-para-el-parametro-beta_1-y-para-el-parametro-beta_4-asumiendo-normalidad-de-los-errores.-contienen-estos-intervalos-a-los-verdaderos-parametros-para-la-muestra-simulada-guarde-en-un-nuevo-objeto-un-uno-si-lo-contiene-y-un-cero-sino-para-cada-uno-de-los-dos-intervalos.}}

\hypertarget{v.-construya-el-intervalo-de-confianza-de-nivel-asintotico-0.90-para-el-parametro-beta_1-y-para-el-parametro-beta_4.-contienen-estos-intervalos-a-los-verdaderos-parametros-para-la-muestra-simulada-guarde-en-un-nuevo-objeto-un-1-si-lo-contiene-y-un-cero-sino-para-cada-uno-de-los-dos-intervalos.}{%
\subsubsection{\texorpdfstring{v. Construya el intervalo de confianza de
nivel asintótico 0.90 para el parámetro \(\beta_1\) y para el parámetro
\(\beta_4\). ¿Contienen estos intervalos a los verdaderos parámetros
para la muestra simulada? Guarde en un nuevo objeto un 1 si lo contiene,
y un cero sino, para cada uno de los dos
intervalos.}{v. Construya el intervalo de confianza de nivel asintótico 0.90 para el parámetro \textbackslash beta\_1 y para el parámetro \textbackslash beta\_4. ¿Contienen estos intervalos a los verdaderos parámetros para la muestra simulada? Guarde en un nuevo objeto un 1 si lo contiene, y un cero sino, para cada uno de los dos intervalos.}}\label{v.-construya-el-intervalo-de-confianza-de-nivel-asintotico-0.90-para-el-parametro-beta_1-y-para-el-parametro-beta_4.-contienen-estos-intervalos-a-los-verdaderos-parametros-para-la-muestra-simulada-guarde-en-un-nuevo-objeto-un-1-si-lo-contiene-y-un-cero-sino-para-cada-uno-de-los-dos-intervalos.}}

\hypertarget{vi.-repita-los-items-ai-hasta-av-b-1000-veces-de-modo-de-tener-una-muestra-de-tamano-b-de-los-estimadores-de-cada-beta_j.-diria-que-la-distribucion-on-de-los-estimadores-de-beta_2-puede-aproximarse-por-la-normal-haga-graficos-que-le-permitan-tomar-esta-decision.-que-proporcion-de-los-b-intervalos-calculados-para-beta_1-y-beta_4-basados-en-una-muestra-de-n-observaciones-contuvo-al-verdadero-valor-del-parametro-responda-para-cada-tipo-de-intervalo-calculado.}{%
\subsubsection{\texorpdfstring{vi. Repita los items a)i) hasta a)v) B =
1000 veces, de modo de tener una muestra de tamaño \(B\) de los
estimadores de cada \(\beta_j\). ¿Diría que la distribución on de los
estimadores de \(\beta_2\) puede aproximarse por la normal? Haga
gráficos que le permitan tomar esta decisión. ¿Qué proporción de los
\(B\) intervalos calculados para \(\beta_1\) y \(\beta_4\) basados en
una muestra de \(n\) observaciones contuvo al verdadero valor del
parámetro? Responda para cada tipo de intervalo
calculado.}{vi. Repita los items a)i) hasta a)v) B = 1000 veces, de modo de tener una muestra de tamaño B de los estimadores de cada \textbackslash beta\_j. ¿Diría que la distribución on de los estimadores de \textbackslash beta\_2 puede aproximarse por la normal? Haga gráficos que le permitan tomar esta decisión. ¿Qué proporción de los B intervalos calculados para \textbackslash beta\_1 y \textbackslash beta\_4 basados en una muestra de n observaciones contuvo al verdadero valor del parámetro? Responda para cada tipo de intervalo calculado.}}\label{vi.-repita-los-items-ai-hasta-av-b-1000-veces-de-modo-de-tener-una-muestra-de-tamano-b-de-los-estimadores-de-cada-beta_j.-diria-que-la-distribucion-on-de-los-estimadores-de-beta_2-puede-aproximarse-por-la-normal-haga-graficos-que-le-permitan-tomar-esta-decision.-que-proporcion-de-los-b-intervalos-calculados-para-beta_1-y-beta_4-basados-en-una-muestra-de-n-observaciones-contuvo-al-verdadero-valor-del-parametro-responda-para-cada-tipo-de-intervalo-calculado.}}

\hypertarget{b-repita-a-para-n25-y-n100.}{%
\subsection{\texorpdfstring{b) Repita \(a\) para \(n=25\) y
\(n=100\).}{b) Repita a para n=25 y n=100.}}\label{b-repita-a-para-n25-y-n100.}}

\hypertarget{c-repita-a-y-b-para-el-caso-de-tener-errores-con-distribucion-lognormalmusigma2-emusigma22-tomando-mu0-y-sigma21.-si-para-alguna-de-las-distribuciones-no-consigue-convencerse-de-que-los-hatbeta-tienen-distribucion-que-puede-ser-aproximada-por-una-normal-repita-para-errores-generados-con-esta-distribucion-en-el-esquenma-de-simulacion-anterior-pero-con-n25050010001500020003000.-exhiba-los-resultados-en-una-tabla-y-comente-brevemente-sus-conclusiones.}{%
\subsection{\texorpdfstring{c) Repita \(a\) y \(b\) para el caso de
tener errores con distribución
\(Lognormal(\mu,\sigma^2)-e^{\mu+{\sigma^2}/2}\), tomando \(\mu=0\) y
\(\sigma^2=1\). Si para alguna de las distribuciones no consigue
convencerse de que los \(\hat{\beta}\) tienen distribución que puede ser
aproximada por una normal, repita para errores generados con esta
distribución en el esquenma de simulación anterior pero con
\(n=250,500,1000,15000,2000,3000\). Exhiba los resultados en una tabla y
comente brevemente sus
conclusiones.}{c) Repita a y b para el caso de tener errores con distribución Lognormal(\textbackslash mu,\textbackslash sigma\^{}2)-e\^{}\{\textbackslash mu+\{\textbackslash sigma\^{}2\}/2\}, tomando \textbackslash mu=0 y \textbackslash sigma\^{}2=1. Si para alguna de las distribuciones no consigue convencerse de que los \textbackslash hat\{\textbackslash beta\} tienen distribución que puede ser aproximada por una normal, repita para errores generados con esta distribución en el esquenma de simulación anterior pero con n=250,500,1000,15000,2000,3000. Exhiba los resultados en una tabla y comente brevemente sus conclusiones.}}\label{c-repita-a-y-b-para-el-caso-de-tener-errores-con-distribucion-lognormalmusigma2-emusigma22-tomando-mu0-y-sigma21.-si-para-alguna-de-las-distribuciones-no-consigue-convencerse-de-que-los-hatbeta-tienen-distribucion-que-puede-ser-aproximada-por-una-normal-repita-para-errores-generados-con-esta-distribucion-en-el-esquenma-de-simulacion-anterior-pero-con-n25050010001500020003000.-exhiba-los-resultados-en-una-tabla-y-comente-brevemente-sus-conclusiones.}}

\hypertarget{d-repita-c-pero-ahora-con-la-distribucion-de-errores-mathcalu-33-y-con-chi2_k--k-con-k3-y-t_k-con-k3.}{%
\subsection{\texorpdfstring{d) Repita \(c\) pero ahora con la
distribución de errores \(\mathcal{U}(-3;3)\) y con \({\chi^2}_k -k\)
con \(k=3\) y \(t_k\) con
\(k=3\).}{d) Repita c pero ahora con la distribución de errores \textbackslash mathcal\{U\}(-3;3) y con \{\textbackslash chi\^{}2\}\_k -k con k=3 y t\_k con k=3.}}\label{d-repita-c-pero-ahora-con-la-distribucion-de-errores-mathcalu-33-y-con-chi2_k--k-con-k3-y-t_k-con-k3.}}

\hypertarget{desarrollo}{%
\section{Desarrollo}\label{desarrollo}}

\hypertarget{metodo-empleado}{%
\subsection{Método Empleado}\label{metodo-empleado}}

Dado que se solicita muchas repeticiones de consignas similares variando
ciertos parámetros, comenzamos por elaborar una gran cantidad de
simulaciones con todas las distribuciones contempladas para el error (y
algunas de nuestra añadidura) con la máxima cantidad de repeticiones y
el máximo número de muestras por simulación. Después vamos a ir
acudiendo a ellas para tomar subconjuntos. Esto lo hicimos trivialmente
y sin remuestrear, tomando los primeros \(n\) elementos de cada
simulación dado que como estas son efectivamente simulaciones de
procesos aleatorios no nos pareció que el remuestreo fuera crítico a
nuestros fines.

Dejamos una serie de parámetros que permitirían generalizar aún más la
simulación, el modelo del proceso generador de datos \texttt{beta\_pgd},
el conjunto de \texttt{n} adoptados, la cantidad de simulaciones
\texttt{n\_sim}, etc.

Dado que esta tabla y la siguiente involucran cantidades nada
despreciables de cálculos, acá adjuntamos una versión del código similar
a la usada pero parametrizada para muchas menos repeticiones, al lado
aparecen comentados los verdaderos vectores que se usaron.

\hypertarget{parametros}{%
\paragraph{Parámetros}\label{parametros}}

\begin{Shaded}
\begin{Highlighting}[]
\KeywordTok{library}\NormalTok{(}\StringTok{'tidyverse'}\NormalTok{)}
\KeywordTok{library}\NormalTok{(}\StringTok{'stats'}\NormalTok{)}
\KeywordTok{library}\NormalTok{(}\StringTok{'future'}\NormalTok{)}
\KeywordTok{library}\NormalTok{(}\StringTok{'furrr'}\NormalTok{)}
\KeywordTok{library}\NormalTok{(}\StringTok{'knitr'}\NormalTok{)}
\KeywordTok{library}\NormalTok{(}\StringTok{'kableExtra'}\NormalTok{)}
\KeywordTok{set.seed}\NormalTok{(}\DecValTok{42}\NormalTok{)}

\CommentTok{# Coeficientes "platonicos" (i.e., del proceso generador de datos)}
\NormalTok{beta_pgd <-}\StringTok{ }\KeywordTok{c}\NormalTok{(}\DecValTok{4}\NormalTok{, }\DecValTok{2}\NormalTok{, }\DecValTok{-3}\NormalTok{, }\FloatTok{0.5}\NormalTok{, }\DecValTok{0}\NormalTok{)}

\NormalTok{metodos_intervalo <-}\StringTok{ }\KeywordTok{c}\NormalTok{(}\StringTok{"asintotico"}\NormalTok{, }\StringTok{"exacto"}\NormalTok{)}
\NormalTok{alfa <-}\StringTok{ }\FloatTok{0.1}

\CommentTok{# Funciones generadoras de x_i}
\NormalTok{generadores_x <-}\StringTok{ }\KeywordTok{list}\NormalTok{(}
    \StringTok{"x1"}\NormalTok{ =}\StringTok{ }\ControlFlowTok{function}\NormalTok{(n) \{ }\KeywordTok{runif}\NormalTok{(n, }\DataTypeTok{min=}\OperatorTok{-}\DecValTok{5}\NormalTok{, }\DataTypeTok{max=}\DecValTok{5}\NormalTok{) \},}
    \StringTok{"x2"}\NormalTok{ =}\StringTok{ }\ControlFlowTok{function}\NormalTok{(n) \{ }\KeywordTok{runif}\NormalTok{(n, }\DataTypeTok{min=}\OperatorTok{-}\DecValTok{5}\NormalTok{, }\DataTypeTok{max=}\DecValTok{5}\NormalTok{) \},}
    \StringTok{"x3"}\NormalTok{ =}\StringTok{ }\ControlFlowTok{function}\NormalTok{(n) \{ }\KeywordTok{runif}\NormalTok{(n, }\DataTypeTok{min=}\OperatorTok{-}\DecValTok{5}\NormalTok{, }\DataTypeTok{max=}\DecValTok{5}\NormalTok{) \},}
    \StringTok{"x4"}\NormalTok{ =}\StringTok{ }\ControlFlowTok{function}\NormalTok{(n) \{ }\KeywordTok{runif}\NormalTok{(n, }\DataTypeTok{min=}\OperatorTok{-}\DecValTok{5}\NormalTok{, }\DataTypeTok{max=}\DecValTok{5}\NormalTok{) \}}
\NormalTok{)}

\NormalTok{generadores_eps <-}\StringTok{ }\KeywordTok{list}\NormalTok{(}
  \StringTok{"normal"}\NormalTok{ =}\StringTok{ }\ControlFlowTok{function}\NormalTok{(n) \{ }\KeywordTok{rnorm}\NormalTok{(n) \},}
  \StringTok{"exponencial"}\NormalTok{ =}\StringTok{ }\ControlFlowTok{function}\NormalTok{(n) \{ }\KeywordTok{rexp}\NormalTok{(n, }\DataTypeTok{rate =} \DecValTok{1}\OperatorTok{/}\DecValTok{2}\NormalTok{) }\OperatorTok{-}\StringTok{ }\DecValTok{2}\NormalTok{ \},}
  \StringTok{"lognormal"}\NormalTok{ =}\StringTok{ }\ControlFlowTok{function}\NormalTok{(n) \{ }\KeywordTok{exp}\NormalTok{(}\KeywordTok{rnorm}\NormalTok{(n) }\OperatorTok{-}\StringTok{ }\KeywordTok{exp}\NormalTok{(}\FloatTok{0.5}\NormalTok{))  \},}
  \StringTok{"uniforme"}\NormalTok{ =}\StringTok{ }\ControlFlowTok{function}\NormalTok{(n) \{ }\KeywordTok{runif}\NormalTok{(n, }\DecValTok{-3}\NormalTok{, }\DecValTok{3}\NormalTok{) \},}
  \StringTok{"chi_cuadrado"}\NormalTok{ =}\StringTok{ }\ControlFlowTok{function}\NormalTok{(n) \{ }\KeywordTok{rchisq}\NormalTok{(n, }\DecValTok{3}\NormalTok{) }\OperatorTok{-}\StringTok{ }\DecValTok{3}\NormalTok{ \},}
  \StringTok{"student1"}\NormalTok{ =}\StringTok{ }\ControlFlowTok{function}\NormalTok{(n) \{ }\KeywordTok{rt}\NormalTok{(n, }\DecValTok{1}\NormalTok{) \},}
  \StringTok{"student3"}\NormalTok{ =}\StringTok{ }\ControlFlowTok{function}\NormalTok{(n) \{ }\KeywordTok{rt}\NormalTok{(n, }\DecValTok{3}\NormalTok{) \}}
\NormalTok{)}

\NormalTok{funciones_a <-}\StringTok{ }\KeywordTok{list}\NormalTok{(}
  \DataTypeTok{beta1 =} \KeywordTok{c}\NormalTok{(}\DecValTok{0}\NormalTok{, }\DecValTok{1}\NormalTok{, }\DecValTok{0}\NormalTok{, }\DecValTok{0}\NormalTok{, }\DecValTok{0}\NormalTok{),}
  \DataTypeTok{beta4 =} \KeywordTok{c}\NormalTok{(}\DecValTok{0}\NormalTok{, }\DecValTok{0}\NormalTok{, }\DecValTok{0}\NormalTok{, }\DecValTok{0}\NormalTok{, }\DecValTok{1}\NormalTok{)}
\NormalTok{)}

\NormalTok{generador_y <-}\StringTok{ }\ControlFlowTok{function}\NormalTok{(x1, x2, x3, x4, beta_pgd, eps, ...)   \{}
  \KeywordTok{c}\NormalTok{(}\DecValTok{1}\NormalTok{, x1, x2, x3, x4) }\OperatorTok\StringTok{ }\NormalTok{beta_pgd }\OperatorTok{+}\StringTok{ }\NormalTok{eps}
\NormalTok{\}}
\end{Highlighting}
\end{Shaded}

\hypertarget{obtencion-de-la-muestra.}{%
\paragraph{Obtención de la muestra.}\label{obtencion-de-la-muestra.}}

Para el vector con todos los valores de \(n\), el conjunto de muestras
pesa 3GB. Por esta razón, optamos por trabajar con otra tabla que acude
a la tabla con las muestras cuando son necesarias y toma los datos
pedidos para hallar los intervalos. Pusimos la salida de ambos tibble
mostrando su forma, \texttt{muestras\_maestras} es el conjunto de las
muestras y \texttt{muestras\_puntuales} conserva sólo la información
necesaria para buscar en la tabla más grande.

\begin{Shaded}
\begin{Highlighting}[]
\NormalTok{generar_muestra <-}\StringTok{ }\ControlFlowTok{function}\NormalTok{(n, generadores_x, generador_eps, beta_pgd) \{}
  \CommentTok{# Tibble vacio}
\NormalTok{  df <-}\StringTok{ }\KeywordTok{tibble}\NormalTok{(}\DataTypeTok{.rows =}\NormalTok{ n)}
  \CommentTok{# Genero variables regresoras y errores}
  \ControlFlowTok{for}\NormalTok{ (nombre }\ControlFlowTok{in} \KeywordTok{names}\NormalTok{(generadores_x)) \{}
    \ControlFlowTok{if}\NormalTok{ (nombre }\OperatorTok{!=}\StringTok{ "y"}\NormalTok{) \{}
\NormalTok{      df[nombre] <-}\StringTok{ }\NormalTok{generadores_x[[nombre]](n)}
\NormalTok{    \}}
\NormalTok{  df}\OperatorTok{$}\NormalTok{eps <-}\StringTok{ }\KeywordTok{generador_eps}\NormalTok{(n)}
\NormalTok{  \}}
  \CommentTok{# Genero y}
\NormalTok{  df[}\StringTok{"y"}\NormalTok{] <-}\StringTok{ }\KeywordTok{pmap_dbl}\NormalTok{(df, generador_y, }\DataTypeTok{beta_pgd=}\NormalTok{beta_pgd)}

  \KeywordTok{return}\NormalTok{(df)}
\NormalTok{\}}

\NormalTok{ayudante_generar_muestra <-}\StringTok{ }\ControlFlowTok{function}\NormalTok{(distr_eps, generadores_x, beta_pgd, n) \{}
  \KeywordTok{generar_muestra}\NormalTok{(n,generadores_x, generadores_eps[[distr_eps]],}\DataTypeTok{beta_pgd=}\NormalTok{beta_pgd)}
\NormalTok{\}}


\NormalTok{n_muestrales <-}\StringTok{ }\KeywordTok{c}\NormalTok{(}\DecValTok{10}\NormalTok{, }\DecValTok{25}\NormalTok{)}
\CommentTok{#n_muestrales <- c(10, 25, 100, 250, 500, 1000, 1500, 2000, 3000)}

\NormalTok{max_n_muestral <-}\StringTok{ }\KeywordTok{max}\NormalTok{(n_muestrales)}
\NormalTok{n_sims <-}\StringTok{ }\DecValTok{1000}
\NormalTok{muestras_maestras <-}\StringTok{ }\KeywordTok{crossing}\NormalTok{(}
  \DataTypeTok{n_sim =} \KeywordTok{seq}\NormalTok{(max_n_muestral),}
  \DataTypeTok{distr_eps =} \KeywordTok{names}\NormalTok{(generadores_eps)) }\OperatorTok
\StringTok{  }\KeywordTok{mutate}\NormalTok{(}
    \DataTypeTok{muestra =} \KeywordTok{future_map}\NormalTok{(}\DataTypeTok{.progress=}\OtherTok{TRUE}\NormalTok{,}
\NormalTok{                  distr_eps,}
\NormalTok{                  ayudante_generar_muestra,}
                  \DataTypeTok{generadores_x =}\NormalTok{ generadores_x,}
                  \DataTypeTok{beta_pgd =}\NormalTok{ beta_pgd,}
                  \DataTypeTok{n =}\NormalTok{ max_n_muestral)}
\NormalTok{  )}

\NormalTok{muestras_maestras}
\end{Highlighting}
\end{Shaded}

\begin{verbatim}
## # A tibble: 175 x 3
##    n_sim distr_eps    muestra          
##    <int> <chr>        <list>           
##  1     1 chi_cuadrado <tibble [25 x 6]>
##  2     1 exponencial  <tibble [25 x 6]>
##  3     1 lognormal    <tibble [25 x 6]>
##  4     1 normal       <tibble [25 x 6]>
##  5     1 student1     <tibble [25 x 6]>
##  6     1 student3     <tibble [25 x 6]>
##  7     1 uniforme     <tibble [25 x 6]>
##  8     2 chi_cuadrado <tibble [25 x 6]>
##  9     2 exponencial  <tibble [25 x 6]>
## 10     2 lognormal    <tibble [25 x 6]>
## # ... with 165 more rows
\end{verbatim}

\begin{Shaded}
\begin{Highlighting}[]
\CommentTok{#El '-3' es poco legible, buscar cómo sustraer una columna por nombre.}

\NormalTok{muestras_puntuales <-}\StringTok{ }\NormalTok{muestras_maestras[}\OperatorTok{-}\DecValTok{3}\NormalTok{] }\OperatorTok
\StringTok{  }\KeywordTok{crossing}\NormalTok{(}
    \DataTypeTok{n =}\NormalTok{ n_muestrales}
\NormalTok{  )}

\NormalTok{muestras_puntuales}
\end{Highlighting}
\end{Shaded}

\begin{verbatim}
## # A tibble: 350 x 3
##    n_sim distr_eps        n
##    <int> <chr>        <dbl>
##  1     1 chi_cuadrado    10
##  2     1 chi_cuadrado    25
##  3     1 exponencial     10
##  4     1 exponencial     25
##  5     1 lognormal       10
##  6     1 lognormal       25
##  7     1 normal          10
##  8     1 normal          25
##  9     1 student1        10
## 10     1 student1        25
## # ... with 340 more rows
\end{verbatim}

\hypertarget{intervalos}{%
\subsection{Intervalos}\label{intervalos}}

Para obtener los intervalos, usamos los parámetros que tiene guardada
cada fila de \texttt{muestras\_puntuales} y ejecutamos una función que
lee en \texttt{muestras\_maestras} de acuerdo a estos.

\begin{Shaded}
\begin{Highlighting}[]
\NormalTok{intervalo_conf <-}\StringTok{ }\ControlFlowTok{function}\NormalTok{(a_vec, llamada_lm, alfa, }\DataTypeTok{metodo =} \StringTok{"exacto"}\NormalTok{) \{}

\NormalTok{  betahat <-}\StringTok{ }\NormalTok{llamada_lm}\OperatorTok{$}\NormalTok{coefficients}
  \CommentTok{# Matriz de covarianza estimada para los coeficientes}
\NormalTok{  Sigmahat <-}\StringTok{ }\KeywordTok{vcov}\NormalTok{(llamada_lm)}

\NormalTok{  n_muestra <-}\StringTok{ }\KeywordTok{nrow}\NormalTok{(llamada_lm}\OperatorTok{$}\NormalTok{model)}
\NormalTok{  r <-}\StringTok{ }\NormalTok{llamada_lm}\OperatorTok{$}\NormalTok{rank}
  \CommentTok{# Cualculo cuantil t o z, segun corresponda}
  \ControlFlowTok{if}\NormalTok{ (metodo }\OperatorTok{==}\StringTok{ "exacto"}\NormalTok{) \{}
\NormalTok{    cuantil <-}\StringTok{ }\KeywordTok{qt}\NormalTok{(}\DataTypeTok{p =} \DecValTok{1} \OperatorTok{-}\StringTok{ }\NormalTok{alfa}\OperatorTok{/}\DecValTok{2}\NormalTok{, }\DataTypeTok{df =}\NormalTok{ n_muestra }\OperatorTok{-}\StringTok{ }\NormalTok{r)}
\NormalTok{  \} }\ControlFlowTok{else} \ControlFlowTok{if}\NormalTok{ (metodo }\OperatorTok{==}\StringTok{ "asintotico"}\NormalTok{) \{}
\NormalTok{    cuantil <-}\StringTok{ }\KeywordTok{qnorm}\NormalTok{(}\DataTypeTok{p =} \DecValTok{1} \OperatorTok{-}\StringTok{ }\NormalTok{alfa}\OperatorTok{/}\DecValTok{2}\NormalTok{)}
\NormalTok{  \} }\ControlFlowTok{else}\NormalTok{ \{}
    \KeywordTok{stop}\NormalTok{(}\StringTok{"Los unicos metodos soportados son 'exacto' y 'asintotico'"}\NormalTok{)}
\NormalTok{  \}}

\NormalTok{  centro <-}\StringTok{ }\KeywordTok{t}\NormalTok{(a_vec)}\OperatorTok\NormalTok{betahat}
\NormalTok{  delta <-}\StringTok{ }\NormalTok{cuantil }\OperatorTok{*}\StringTok{ }\KeywordTok{sqrt}\NormalTok{(}\KeywordTok{t}\NormalTok{(a_vec) }\OperatorTok\StringTok{ }\NormalTok{Sigmahat }\OperatorTok\StringTok{ }\NormalTok{a_vec)}
  \KeywordTok{return}\NormalTok{(}\KeywordTok{c}\NormalTok{(centro }\OperatorTok{-}\StringTok{ }\NormalTok{delta, centro }\OperatorTok{+}\StringTok{ }\NormalTok{delta))}
\NormalTok{\}}

\NormalTok{cubre <-}\StringTok{ }\ControlFlowTok{function}\NormalTok{(intervalo, valor) \{ intervalo[}\DecValTok{1}\NormalTok{] }\OperatorTok{<=}\StringTok{ }\NormalTok{valor }\OperatorTok{&}\StringTok{ }\NormalTok{intervalo[}\DecValTok{2}\NormalTok{] }\OperatorTok{>=}\StringTok{ }\NormalTok{valor\}}

\NormalTok{ayudante_intervalo_conf <-}\StringTok{ }\ControlFlowTok{function}\NormalTok{(n_simulacion, distr_epsilon, n, fun_a, met_int, alfa) \{}
\NormalTok{  muestra_a_evaluar <-}\StringTok{ }\NormalTok{(muestras_maestras }\OperatorTok\StringTok{ }\KeywordTok{filter}\NormalTok{(n_sim}\OperatorTok{==}\NormalTok{n_simulacion,distr_eps}\OperatorTok{==}\NormalTok{distr_epsilon))[[}\DecValTok{1}\NormalTok{,}\StringTok{'muestra'}\NormalTok{]] }\OperatorTok\StringTok{ }\KeywordTok{head}\NormalTok{(n)}
\NormalTok{  modelo <-}\StringTok{ }\KeywordTok{lm}\NormalTok{(y }\OperatorTok{~}\StringTok{ }\NormalTok{x1 }\OperatorTok{+}\StringTok{ }\NormalTok{x2 }\OperatorTok{+}\StringTok{ }\NormalTok{x3 }\OperatorTok{+}\NormalTok{x4,}\DataTypeTok{data=}\NormalTok{muestra_a_evaluar)}
  \KeywordTok{intervalo_conf}\NormalTok{(}\DataTypeTok{a_vec =}\NormalTok{ funciones_a[[fun_a]], }\DataTypeTok{llamada_lm=}\NormalTok{modelo, }\DataTypeTok{alfa=}\NormalTok{alfa, }\DataTypeTok{metodo =}\NormalTok{ met_int)}
\NormalTok{\}}

\NormalTok{intervalos <-}\StringTok{ }\NormalTok{muestras_puntuales }\OperatorTok
\StringTok{  }\KeywordTok{crossing}\NormalTok{(}
    \DataTypeTok{fun_a =} \KeywordTok{names}\NormalTok{(funciones_a),}
    \DataTypeTok{met_int =}\NormalTok{ metodos_intervalo) }\OperatorTok
\StringTok{  }\KeywordTok{mutate}\NormalTok{(}
    \CommentTok{#atbeta es el valor del parámetro en el PGD.}
    \DataTypeTok{atbeta =} \KeywordTok{map_dbl}\NormalTok{(fun_a, }\ControlFlowTok{function}\NormalTok{(i) funciones_a[[i]] }\OperatorTok\StringTok{ }\NormalTok{beta_pgd),}
    \DataTypeTok{ic =} \KeywordTok{future_pmap}\NormalTok{( }\DataTypeTok{.progress =} \OtherTok{TRUE}\NormalTok{,}
      \KeywordTok{list}\NormalTok{(n_sim, distr_eps, n, fun_a, met_int),}
\NormalTok{      ayudante_intervalo_conf,}
      \DataTypeTok{alfa =}\NormalTok{ alfa),}
    \DataTypeTok{cubre =} \KeywordTok{map2_lgl}\NormalTok{(ic, atbeta, cubre),}
    \DataTypeTok{ic_low =} \KeywordTok{map_dbl}\NormalTok{(ic, }\DecValTok{1}\NormalTok{),}
    \DataTypeTok{ic_upp =} \KeywordTok{map_dbl}\NormalTok{(ic, }\DecValTok{2}\NormalTok{)}
\NormalTok{    )}
\end{Highlighting}
\end{Shaded}

\hypertarget{respuestas}{%
\section{Respuestas}\label{respuestas}}

\begin{Shaded}
\begin{Highlighting}[]
\NormalTok{intervalos <-}\StringTok{ }\KeywordTok{read_rds}\NormalTok{(}\StringTok{"simulacion.Rds"}\NormalTok{) }\OperatorTok
\StringTok{  }\KeywordTok{mutate}\NormalTok{(}
    \DataTypeTok{estimador =}\NormalTok{ (ic_upp}\OperatorTok{+}\NormalTok{ic_low)}\OperatorTok{/}\DecValTok{2}
\NormalTok{  )}
\end{Highlighting}
\end{Shaded}

\hypertarget{los-intervalos-cubren-a-los-parametros}{%
\subsection{¿Los intervalos Cubren a los
Parámetros?}\label{los-intervalos-cubren-a-los-parametros}}

Respondemos directamente para todas las distribuciones estudiadas.

\hypertarget{para-b1-y-es-decir-si-el-primer-intento-de-obtener-el-intervalo-cubre-el-valor.}{%
\subsubsection{\texorpdfstring{Para \(B=1\) y, es decir si el primer
intento de obtener el intervalo cubre el
valor.}{Para B=1 y, es decir si el primer intento de obtener el intervalo cubre el valor.}}\label{para-b1-y-es-decir-si-el-primer-intento-de-obtener-el-intervalo-cubre-el-valor.}}

\begin{Shaded}
\begin{Highlighting}[]
\NormalTok{sintesis <-}\StringTok{ }\NormalTok{intervalos }\OperatorTok
\StringTok{  }\KeywordTok{filter}\NormalTok{(n_sim}\OperatorTok{==}\DecValTok{1}\NormalTok{) }\OperatorTok
\StringTok{  }\KeywordTok{group_by}\NormalTok{(distr_eps, n, met_int, fun_a) }\OperatorTok
\StringTok{  }\KeywordTok{summarise}\NormalTok{(}\DataTypeTok{prop_cubre =} \KeywordTok{mean}\NormalTok{(cubre))}

\NormalTok{sintesis }\OperatorTok
\StringTok{  }\KeywordTok{mutate}\NormalTok{ (}
    \DataTypeTok{prop_cubre =} \KeywordTok{round}\NormalTok{(}\DataTypeTok{digits=}\DecValTok{4}\NormalTok{,}\DataTypeTok{x=}\NormalTok{prop_cubre),}
    \DataTypeTok{prop_cubre =} \KeywordTok{cell_spec}\NormalTok{(prop_cubre,}\StringTok{"latex"}\NormalTok{,}
                           \DataTypeTok{color =} \KeywordTok{ifelse}\NormalTok{(prop_cubre }\OperatorTok{<}\StringTok{ }\FloatTok{0.89}\NormalTok{,}\StringTok{"red"}\NormalTok{,}\StringTok{"blue"}\NormalTok{),}
                           \DataTypeTok{background =} \KeywordTok{ifelse}\NormalTok{(prop_cubre }\OperatorTok{<}\StringTok{ }\FloatTok{0.86}\NormalTok{,}\StringTok{"gray"}\NormalTok{,}\StringTok{"white"}\NormalTok{)}
\NormalTok{                           )}
\NormalTok{  ) }\OperatorTok
\StringTok{  }\KeywordTok{spread}\NormalTok{(n,prop_cubre) }\OperatorTok
\StringTok{  }\KeywordTok{kable}\NormalTok{(}\DataTypeTok{format=}\StringTok{"latex"}\NormalTok{, }\DataTypeTok{escape =}\NormalTok{ F ) }\OperatorTok
\StringTok{  }\KeywordTok{kable_styling}\NormalTok{(}\StringTok{"condensed"}\NormalTok{, }\StringTok{"striped"}\NormalTok{) }\OperatorTok
\StringTok{  }\KeywordTok{add_header_above}\NormalTok{(}\KeywordTok{c}\NormalTok{(}\StringTok{" "}\NormalTok{=}\DecValTok{3}\NormalTok{,}\StringTok{"valores de n"}\NormalTok{=}\DecValTok{9}\NormalTok{))}
\end{Highlighting}
\end{Shaded}

\begin{table}[H]
\centering
\begin{tabular}{l|l|l|l|l|l|l|l|l|l|l|l}
\hline
\multicolumn{3}{c|}{ } & \multicolumn{9}{c}{valores de n} \\
\cline{4-12}
distr_eps & met_int & fun_a & 10 & 25 & 100 & 250 & 500 & 1000 & 1500 & 2000 & 3000\\
\hline
\rowcolor{gray!6}  chi_cuadrado & asintotico & beta1 & \cellcolor{gray}{\textcolor{red}{0}} & \cellcolor{white}{\textcolor{blue}{1}} & \cellcolor{white}{\textcolor{blue}{1}} & \cellcolor{white}{\textcolor{blue}{1}} & \cellcolor{white}{\textcolor{blue}{1}} & \cellcolor{white}{\textcolor{blue}{1}} & \cellcolor{white}{\textcolor{blue}{1}} & \cellcolor{gray}{\textcolor{red}{0}} & \cellcolor{gray}{\textcolor{red}{0}}\\
\hline
chi_cuadrado & asintotico & beta4 & \cellcolor{white}{\textcolor{blue}{1}} & \cellcolor{white}{\textcolor{blue}{1}} & \cellcolor{white}{\textcolor{blue}{1}} & \cellcolor{white}{\textcolor{blue}{1}} & \cellcolor{white}{\textcolor{blue}{1}} & \cellcolor{white}{\textcolor{blue}{1}} & \cellcolor{white}{\textcolor{blue}{1}} & \cellcolor{white}{\textcolor{blue}{1}} & \cellcolor{gray}{\textcolor{red}{0}}\\
\hline
\rowcolor{gray!6}  chi_cuadrado & exacto & beta1 & \cellcolor{gray}{\textcolor{red}{0}} & \cellcolor{white}{\textcolor{blue}{1}} & \cellcolor{white}{\textcolor{blue}{1}} & \cellcolor{white}{\textcolor{blue}{1}} & \cellcolor{white}{\textcolor{blue}{1}} & \cellcolor{white}{\textcolor{blue}{1}} & \cellcolor{white}{\textcolor{blue}{1}} & \cellcolor{gray}{\textcolor{red}{0}} & \cellcolor{gray}{\textcolor{red}{0}}\\
\hline
chi_cuadrado & exacto & beta4 & \cellcolor{white}{\textcolor{blue}{1}} & \cellcolor{white}{\textcolor{blue}{1}} & \cellcolor{white}{\textcolor{blue}{1}} & \cellcolor{white}{\textcolor{blue}{1}} & \cellcolor{white}{\textcolor{blue}{1}} & \cellcolor{white}{\textcolor{blue}{1}} & \cellcolor{white}{\textcolor{blue}{1}} & \cellcolor{white}{\textcolor{blue}{1}} & \cellcolor{gray}{\textcolor{red}{0}}\\
\hline
\rowcolor{gray!6}  exponencial & asintotico & beta1 & \cellcolor{white}{\textcolor{blue}{1}} & \cellcolor{white}{\textcolor{blue}{1}} & \cellcolor{white}{\textcolor{blue}{1}} & \cellcolor{white}{\textcolor{blue}{1}} & \cellcolor{gray}{\textcolor{red}{0}} & \cellcolor{white}{\textcolor{blue}{1}} & \cellcolor{white}{\textcolor{blue}{1}} & \cellcolor{white}{\textcolor{blue}{1}} & \cellcolor{white}{\textcolor{blue}{1}}\\
\hline
exponencial & asintotico & beta4 & \cellcolor{white}{\textcolor{blue}{1}} & \cellcolor{white}{\textcolor{blue}{1}} & \cellcolor{white}{\textcolor{blue}{1}} & \cellcolor{white}{\textcolor{blue}{1}} & \cellcolor{white}{\textcolor{blue}{1}} & \cellcolor{white}{\textcolor{blue}{1}} & \cellcolor{white}{\textcolor{blue}{1}} & \cellcolor{white}{\textcolor{blue}{1}} & \cellcolor{white}{\textcolor{blue}{1}}\\
\hline
\rowcolor{gray!6}  exponencial & exacto & beta1 & \cellcolor{white}{\textcolor{blue}{1}} & \cellcolor{white}{\textcolor{blue}{1}} & \cellcolor{white}{\textcolor{blue}{1}} & \cellcolor{white}{\textcolor{blue}{1}} & \cellcolor{gray}{\textcolor{red}{0}} & \cellcolor{white}{\textcolor{blue}{1}} & \cellcolor{white}{\textcolor{blue}{1}} & \cellcolor{white}{\textcolor{blue}{1}} & \cellcolor{white}{\textcolor{blue}{1}}\\
\hline
exponencial & exacto & beta4 & \cellcolor{white}{\textcolor{blue}{1}} & \cellcolor{white}{\textcolor{blue}{1}} & \cellcolor{white}{\textcolor{blue}{1}} & \cellcolor{white}{\textcolor{blue}{1}} & \cellcolor{white}{\textcolor{blue}{1}} & \cellcolor{white}{\textcolor{blue}{1}} & \cellcolor{white}{\textcolor{blue}{1}} & \cellcolor{white}{\textcolor{blue}{1}} & \cellcolor{white}{\textcolor{blue}{1}}\\
\hline
\rowcolor{gray!6}  lognormal & asintotico & beta1 & \cellcolor{white}{\textcolor{blue}{1}} & \cellcolor{white}{\textcolor{blue}{1}} & \cellcolor{white}{\textcolor{blue}{1}} & \cellcolor{white}{\textcolor{blue}{1}} & \cellcolor{white}{\textcolor{blue}{1}} & \cellcolor{white}{\textcolor{blue}{1}} & \cellcolor{white}{\textcolor{blue}{1}} & \cellcolor{white}{\textcolor{blue}{1}} & \cellcolor{gray}{\textcolor{red}{0}}\\
\hline
lognormal & asintotico & beta4 & \cellcolor{white}{\textcolor{blue}{1}} & \cellcolor{white}{\textcolor{blue}{1}} & \cellcolor{white}{\textcolor{blue}{1}} & \cellcolor{white}{\textcolor{blue}{1}} & \cellcolor{gray}{\textcolor{red}{0}} & \cellcolor{white}{\textcolor{blue}{1}} & \cellcolor{gray}{\textcolor{red}{0}} & \cellcolor{white}{\textcolor{blue}{1}} & \cellcolor{white}{\textcolor{blue}{1}}\\
\hline
\rowcolor{gray!6}  lognormal & exacto & beta1 & \cellcolor{white}{\textcolor{blue}{1}} & \cellcolor{white}{\textcolor{blue}{1}} & \cellcolor{white}{\textcolor{blue}{1}} & \cellcolor{white}{\textcolor{blue}{1}} & \cellcolor{white}{\textcolor{blue}{1}} & \cellcolor{white}{\textcolor{blue}{1}} & \cellcolor{white}{\textcolor{blue}{1}} & \cellcolor{white}{\textcolor{blue}{1}} & \cellcolor{gray}{\textcolor{red}{0}}\\
\hline
lognormal & exacto & beta4 & \cellcolor{white}{\textcolor{blue}{1}} & \cellcolor{white}{\textcolor{blue}{1}} & \cellcolor{white}{\textcolor{blue}{1}} & \cellcolor{white}{\textcolor{blue}{1}} & \cellcolor{gray}{\textcolor{red}{0}} & \cellcolor{white}{\textcolor{blue}{1}} & \cellcolor{gray}{\textcolor{red}{0}} & \cellcolor{white}{\textcolor{blue}{1}} & \cellcolor{white}{\textcolor{blue}{1}}\\
\hline
\rowcolor{gray!6}  normal & asintotico & beta1 & \cellcolor{white}{\textcolor{blue}{1}} & \cellcolor{white}{\textcolor{blue}{1}} & \cellcolor{white}{\textcolor{blue}{1}} & \cellcolor{white}{\textcolor{blue}{1}} & \cellcolor{white}{\textcolor{blue}{1}} & \cellcolor{gray}{\textcolor{red}{0}} & \cellcolor{white}{\textcolor{blue}{1}} & \cellcolor{white}{\textcolor{blue}{1}} & \cellcolor{white}{\textcolor{blue}{1}}\\
\hline
normal & asintotico & beta4 & \cellcolor{gray}{\textcolor{red}{0}} & \cellcolor{white}{\textcolor{blue}{1}} & \cellcolor{white}{\textcolor{blue}{1}} & \cellcolor{white}{\textcolor{blue}{1}} & \cellcolor{gray}{\textcolor{red}{0}} & \cellcolor{gray}{\textcolor{red}{0}} & \cellcolor{gray}{\textcolor{red}{0}} & \cellcolor{white}{\textcolor{blue}{1}} & \cellcolor{white}{\textcolor{blue}{1}}\\
\hline
\rowcolor{gray!6}  normal & exacto & beta1 & \cellcolor{white}{\textcolor{blue}{1}} & \cellcolor{white}{\textcolor{blue}{1}} & \cellcolor{white}{\textcolor{blue}{1}} & \cellcolor{white}{\textcolor{blue}{1}} & \cellcolor{white}{\textcolor{blue}{1}} & \cellcolor{gray}{\textcolor{red}{0}} & \cellcolor{white}{\textcolor{blue}{1}} & \cellcolor{white}{\textcolor{blue}{1}} & \cellcolor{white}{\textcolor{blue}{1}}\\
\hline
normal & exacto & beta4 & \cellcolor{gray}{\textcolor{red}{0}} & \cellcolor{white}{\textcolor{blue}{1}} & \cellcolor{white}{\textcolor{blue}{1}} & \cellcolor{white}{\textcolor{blue}{1}} & \cellcolor{gray}{\textcolor{red}{0}} & \cellcolor{gray}{\textcolor{red}{0}} & \cellcolor{gray}{\textcolor{red}{0}} & \cellcolor{white}{\textcolor{blue}{1}} & \cellcolor{white}{\textcolor{blue}{1}}\\
\hline
\rowcolor{gray!6}  student1 & asintotico & beta1 & \cellcolor{white}{\textcolor{blue}{1}} & \cellcolor{white}{\textcolor{blue}{1}} & \cellcolor{white}{\textcolor{blue}{1}} & \cellcolor{white}{\textcolor{blue}{1}} & \cellcolor{white}{\textcolor{blue}{1}} & \cellcolor{white}{\textcolor{blue}{1}} & \cellcolor{white}{\textcolor{blue}{1}} & \cellcolor{white}{\textcolor{blue}{1}} & \cellcolor{white}{\textcolor{blue}{1}}\\
\hline
student1 & asintotico & beta4 & \cellcolor{white}{\textcolor{blue}{1}} & \cellcolor{white}{\textcolor{blue}{1}} & \cellcolor{white}{\textcolor{blue}{1}} & \cellcolor{white}{\textcolor{blue}{1}} & \cellcolor{white}{\textcolor{blue}{1}} & \cellcolor{white}{\textcolor{blue}{1}} & \cellcolor{white}{\textcolor{blue}{1}} & \cellcolor{white}{\textcolor{blue}{1}} & \cellcolor{white}{\textcolor{blue}{1}}\\
\hline
\rowcolor{gray!6}  student1 & exacto & beta1 & \cellcolor{white}{\textcolor{blue}{1}} & \cellcolor{white}{\textcolor{blue}{1}} & \cellcolor{white}{\textcolor{blue}{1}} & \cellcolor{white}{\textcolor{blue}{1}} & \cellcolor{white}{\textcolor{blue}{1}} & \cellcolor{white}{\textcolor{blue}{1}} & \cellcolor{white}{\textcolor{blue}{1}} & \cellcolor{white}{\textcolor{blue}{1}} & \cellcolor{white}{\textcolor{blue}{1}}\\
\hline
student1 & exacto & beta4 & \cellcolor{white}{\textcolor{blue}{1}} & \cellcolor{white}{\textcolor{blue}{1}} & \cellcolor{white}{\textcolor{blue}{1}} & \cellcolor{white}{\textcolor{blue}{1}} & \cellcolor{white}{\textcolor{blue}{1}} & \cellcolor{white}{\textcolor{blue}{1}} & \cellcolor{white}{\textcolor{blue}{1}} & \cellcolor{white}{\textcolor{blue}{1}} & \cellcolor{white}{\textcolor{blue}{1}}\\
\hline
\rowcolor{gray!6}  student3 & asintotico & beta1 & \cellcolor{white}{\textcolor{blue}{1}} & \cellcolor{white}{\textcolor{blue}{1}} & \cellcolor{white}{\textcolor{blue}{1}} & \cellcolor{white}{\textcolor{blue}{1}} & \cellcolor{white}{\textcolor{blue}{1}} & \cellcolor{white}{\textcolor{blue}{1}} & \cellcolor{white}{\textcolor{blue}{1}} & \cellcolor{white}{\textcolor{blue}{1}} & \cellcolor{white}{\textcolor{blue}{1}}\\
\hline
student3 & asintotico & beta4 & \cellcolor{gray}{\textcolor{red}{0}} & \cellcolor{white}{\textcolor{blue}{1}} & \cellcolor{white}{\textcolor{blue}{1}} & \cellcolor{gray}{\textcolor{red}{0}} & \cellcolor{white}{\textcolor{blue}{1}} & \cellcolor{white}{\textcolor{blue}{1}} & \cellcolor{white}{\textcolor{blue}{1}} & \cellcolor{white}{\textcolor{blue}{1}} & \cellcolor{white}{\textcolor{blue}{1}}\\
\hline
\rowcolor{gray!6}  student3 & exacto & beta1 & \cellcolor{white}{\textcolor{blue}{1}} & \cellcolor{white}{\textcolor{blue}{1}} & \cellcolor{white}{\textcolor{blue}{1}} & \cellcolor{white}{\textcolor{blue}{1}} & \cellcolor{white}{\textcolor{blue}{1}} & \cellcolor{white}{\textcolor{blue}{1}} & \cellcolor{white}{\textcolor{blue}{1}} & \cellcolor{white}{\textcolor{blue}{1}} & \cellcolor{white}{\textcolor{blue}{1}}\\
\hline
student3 & exacto & beta4 & \cellcolor{gray}{\textcolor{red}{0}} & \cellcolor{white}{\textcolor{blue}{1}} & \cellcolor{white}{\textcolor{blue}{1}} & \cellcolor{gray}{\textcolor{red}{0}} & \cellcolor{white}{\textcolor{blue}{1}} & \cellcolor{white}{\textcolor{blue}{1}} & \cellcolor{white}{\textcolor{blue}{1}} & \cellcolor{white}{\textcolor{blue}{1}} & \cellcolor{white}{\textcolor{blue}{1}}\\
\hline
\rowcolor{gray!6}  uniforme & asintotico & beta1 & \cellcolor{white}{\textcolor{blue}{1}} & \cellcolor{white}{\textcolor{blue}{1}} & \cellcolor{white}{\textcolor{blue}{1}} & \cellcolor{white}{\textcolor{blue}{1}} & \cellcolor{white}{\textcolor{blue}{1}} & \cellcolor{white}{\textcolor{blue}{1}} & \cellcolor{white}{\textcolor{blue}{1}} & \cellcolor{white}{\textcolor{blue}{1}} & \cellcolor{white}{\textcolor{blue}{1}}\\
\hline
uniforme & asintotico & beta4 & \cellcolor{white}{\textcolor{blue}{1}} & \cellcolor{white}{\textcolor{blue}{1}} & \cellcolor{white}{\textcolor{blue}{1}} & \cellcolor{white}{\textcolor{blue}{1}} & \cellcolor{white}{\textcolor{blue}{1}} & \cellcolor{white}{\textcolor{blue}{1}} & \cellcolor{white}{\textcolor{blue}{1}} & \cellcolor{white}{\textcolor{blue}{1}} & \cellcolor{white}{\textcolor{blue}{1}}\\
\hline
\rowcolor{gray!6}  uniforme & exacto & beta1 & \cellcolor{white}{\textcolor{blue}{1}} & \cellcolor{white}{\textcolor{blue}{1}} & \cellcolor{white}{\textcolor{blue}{1}} & \cellcolor{white}{\textcolor{blue}{1}} & \cellcolor{white}{\textcolor{blue}{1}} & \cellcolor{white}{\textcolor{blue}{1}} & \cellcolor{white}{\textcolor{blue}{1}} & \cellcolor{white}{\textcolor{blue}{1}} & \cellcolor{white}{\textcolor{blue}{1}}\\
\hline
uniforme & exacto & beta4 & \cellcolor{white}{\textcolor{blue}{1}} & \cellcolor{white}{\textcolor{blue}{1}} & \cellcolor{white}{\textcolor{blue}{1}} & \cellcolor{white}{\textcolor{blue}{1}} & \cellcolor{white}{\textcolor{blue}{1}} & \cellcolor{white}{\textcolor{blue}{1}} & \cellcolor{white}{\textcolor{blue}{1}} & \cellcolor{white}{\textcolor{blue}{1}} & \cellcolor{white}{\textcolor{blue}{1}}\\
\hline
\end{tabular}
\end{table}

\hypertarget{para-b1000}{%
\subsubsection{\texorpdfstring{Para
\(B=1000\)}{Para B=1000}}\label{para-b1000}}

\begin{Shaded}
\begin{Highlighting}[]
\NormalTok{sintesis <-}\StringTok{ }\NormalTok{intervalos }\OperatorTok
\StringTok{  }\KeywordTok{filter}\NormalTok{(n_sim}\OperatorTok{<=}\DecValTok{1000}\NormalTok{) }\OperatorTok
\StringTok{  }\KeywordTok{group_by}\NormalTok{(distr_eps, n, met_int, fun_a) }\OperatorTok
\StringTok{  }\KeywordTok{summarise}\NormalTok{(}\DataTypeTok{prop_cubre =} \KeywordTok{mean}\NormalTok{(cubre))}

\NormalTok{sintesis }\OperatorTok
\StringTok{  }\KeywordTok{mutate}\NormalTok{ (}
    \DataTypeTok{prop_cubre =} \KeywordTok{round}\NormalTok{(}\DataTypeTok{digits=}\DecValTok{4}\NormalTok{,}\DataTypeTok{x=}\NormalTok{prop_cubre),}
    \DataTypeTok{prop_cubre =} \KeywordTok{cell_spec}\NormalTok{(prop_cubre,}\StringTok{"latex"}\NormalTok{,}
                           \DataTypeTok{color =} \KeywordTok{ifelse}\NormalTok{(prop_cubre }\OperatorTok{<}\StringTok{ }\FloatTok{0.89}\NormalTok{,}\StringTok{"red"}\NormalTok{,}\StringTok{"blue"}\NormalTok{),}
                           \DataTypeTok{background =} \KeywordTok{ifelse}\NormalTok{(prop_cubre }\OperatorTok{<}\StringTok{ }\FloatTok{0.86}\NormalTok{,}\StringTok{"gray"}\NormalTok{,}\StringTok{"white"}\NormalTok{)}
\NormalTok{                           )}
\NormalTok{  ) }\OperatorTok
\StringTok{  }\KeywordTok{spread}\NormalTok{(n,prop_cubre) }\OperatorTok
\StringTok{  }\KeywordTok{kable}\NormalTok{(}\DataTypeTok{format=}\StringTok{"latex"}\NormalTok{, }\DataTypeTok{escape =}\NormalTok{ F) }\OperatorTok
\StringTok{  }\KeywordTok{kable_styling}\NormalTok{(}\StringTok{"condensed"}\NormalTok{, }\StringTok{"striped"}\NormalTok{) }\OperatorTok
\StringTok{  }\KeywordTok{add_header_above}\NormalTok{(}\KeywordTok{c}\NormalTok{(}\StringTok{" "}\NormalTok{=}\DecValTok{3}\NormalTok{,}\StringTok{"valores de n"}\NormalTok{=}\DecValTok{9}\NormalTok{))}
\end{Highlighting}
\end{Shaded}

\begin{table}[H]
\centering
\begin{tabular}{l|l|l|l|l|l|l|l|l|l|l|l}
\hline
\multicolumn{3}{c|}{ } & \multicolumn{9}{c}{valores de n} \\
\cline{4-12}
distr_eps & met_int & fun_a & 10 & 25 & 100 & 250 & 500 & 1000 & 1500 & 2000 & 3000\\
\hline
\rowcolor{gray!6}  chi_cuadrado & asintotico & beta1 & \cellcolor{gray}{\textcolor{red}{0.841}} & \cellcolor{white}{\textcolor{blue}{0.909}} & \cellcolor{white}{\textcolor{blue}{0.906}} & \cellcolor{white}{\textcolor{blue}{0.893}} & \cellcolor{white}{\textcolor{blue}{0.895}} & \cellcolor{white}{\textcolor{blue}{0.891}} & \cellcolor{white}{\textcolor{red}{0.887}} & \cellcolor{white}{\textcolor{red}{0.889}} & \cellcolor{white}{\textcolor{blue}{0.891}}\\
\hline
chi_cuadrado & asintotico & beta4 & \cellcolor{gray}{\textcolor{red}{0.848}} & \cellcolor{white}{\textcolor{red}{0.884}} & \cellcolor{white}{\textcolor{blue}{0.895}} & \cellcolor{white}{\textcolor{blue}{0.897}} & \cellcolor{white}{\textcolor{blue}{0.907}} & \cellcolor{white}{\textcolor{blue}{0.91}} & \cellcolor{white}{\textcolor{blue}{0.917}} & \cellcolor{white}{\textcolor{blue}{0.919}} & \cellcolor{white}{\textcolor{blue}{0.92}}\\
\hline
\rowcolor{gray!6}  chi_cuadrado & exacto & beta1 & \cellcolor{white}{\textcolor{blue}{0.905}} & \cellcolor{white}{\textcolor{blue}{0.926}} & \cellcolor{white}{\textcolor{blue}{0.908}} & \cellcolor{white}{\textcolor{blue}{0.895}} & \cellcolor{white}{\textcolor{blue}{0.896}} & \cellcolor{white}{\textcolor{blue}{0.891}} & \cellcolor{white}{\textcolor{red}{0.887}} & \cellcolor{white}{\textcolor{blue}{0.89}} & \cellcolor{white}{\textcolor{blue}{0.891}}\\
\hline
chi_cuadrado & exacto & beta4 & \cellcolor{white}{\textcolor{blue}{0.91}} & \cellcolor{white}{\textcolor{blue}{0.898}} & \cellcolor{white}{\textcolor{blue}{0.898}} & \cellcolor{white}{\textcolor{blue}{0.898}} & \cellcolor{white}{\textcolor{blue}{0.908}} & \cellcolor{white}{\textcolor{blue}{0.91}} & \cellcolor{white}{\textcolor{blue}{0.917}} & \cellcolor{white}{\textcolor{blue}{0.919}} & \cellcolor{white}{\textcolor{blue}{0.92}}\\
\hline
\rowcolor{gray!6}  exponencial & asintotico & beta1 & \cellcolor{white}{\textcolor{red}{0.86}} & \cellcolor{white}{\textcolor{red}{0.862}} & \cellcolor{white}{\textcolor{blue}{0.892}} & \cellcolor{white}{\textcolor{blue}{0.907}} & \cellcolor{white}{\textcolor{blue}{0.892}} & \cellcolor{white}{\textcolor{red}{0.879}} & \cellcolor{white}{\textcolor{red}{0.881}} & \cellcolor{white}{\textcolor{red}{0.882}} & \cellcolor{white}{\textcolor{red}{0.869}}\\
\hline
exponencial & asintotico & beta4 & \cellcolor{white}{\textcolor{red}{0.865}} & \cellcolor{white}{\textcolor{blue}{0.89}} & \cellcolor{white}{\textcolor{blue}{0.909}} & \cellcolor{white}{\textcolor{blue}{0.907}} & \cellcolor{white}{\textcolor{blue}{0.911}} & \cellcolor{white}{\textcolor{blue}{0.903}} & \cellcolor{white}{\textcolor{blue}{0.895}} & \cellcolor{white}{\textcolor{blue}{0.894}} & \cellcolor{white}{\textcolor{blue}{0.904}}\\
\hline
\rowcolor{gray!6}  exponencial & exacto & beta1 & \cellcolor{white}{\textcolor{blue}{0.917}} & \cellcolor{white}{\textcolor{red}{0.881}} & \cellcolor{white}{\textcolor{blue}{0.897}} & \cellcolor{white}{\textcolor{blue}{0.907}} & \cellcolor{white}{\textcolor{blue}{0.893}} & \cellcolor{white}{\textcolor{red}{0.88}} & \cellcolor{white}{\textcolor{red}{0.881}} & \cellcolor{white}{\textcolor{red}{0.883}} & \cellcolor{white}{\textcolor{red}{0.869}}\\
\hline
exponencial & exacto & beta4 & \cellcolor{white}{\textcolor{blue}{0.915}} & \cellcolor{white}{\textcolor{blue}{0.905}} & \cellcolor{white}{\textcolor{blue}{0.915}} & \cellcolor{white}{\textcolor{blue}{0.907}} & \cellcolor{white}{\textcolor{blue}{0.911}} & \cellcolor{white}{\textcolor{blue}{0.903}} & \cellcolor{white}{\textcolor{blue}{0.895}} & \cellcolor{white}{\textcolor{blue}{0.895}} & \cellcolor{white}{\textcolor{blue}{0.904}}\\
\hline
\rowcolor{gray!6}  lognormal & asintotico & beta1 & \cellcolor{gray}{\textcolor{red}{0.842}} & \cellcolor{white}{\textcolor{red}{0.888}} & \cellcolor{white}{\textcolor{blue}{0.899}} & \cellcolor{white}{\textcolor{blue}{0.901}} & \cellcolor{white}{\textcolor{red}{0.885}} & \cellcolor{white}{\textcolor{red}{0.885}} & \cellcolor{white}{\textcolor{blue}{0.898}} & \cellcolor{white}{\textcolor{blue}{0.891}} & \cellcolor{white}{\textcolor{blue}{0.903}}\\
\hline
lognormal & asintotico & beta4 & \cellcolor{gray}{\textcolor{red}{0.825}} & \cellcolor{white}{\textcolor{blue}{0.898}} & \cellcolor{white}{\textcolor{blue}{0.917}} & \cellcolor{white}{\textcolor{blue}{0.896}} & \cellcolor{white}{\textcolor{red}{0.884}} & \cellcolor{white}{\textcolor{blue}{0.895}} & \cellcolor{white}{\textcolor{red}{0.889}} & \cellcolor{white}{\textcolor{red}{0.881}} & \cellcolor{white}{\textcolor{blue}{0.894}}\\
\hline
\rowcolor{gray!6}  lognormal & exacto & beta1 & \cellcolor{white}{\textcolor{blue}{0.893}} & \cellcolor{white}{\textcolor{blue}{0.906}} & \cellcolor{white}{\textcolor{blue}{0.902}} & \cellcolor{white}{\textcolor{blue}{0.902}} & \cellcolor{white}{\textcolor{red}{0.885}} & \cellcolor{white}{\textcolor{red}{0.885}} & \cellcolor{white}{\textcolor{blue}{0.898}} & \cellcolor{white}{\textcolor{blue}{0.891}} & \cellcolor{white}{\textcolor{blue}{0.903}}\\
\hline
lognormal & exacto & beta4 & \cellcolor{white}{\textcolor{blue}{0.89}} & \cellcolor{white}{\textcolor{blue}{0.912}} & \cellcolor{white}{\textcolor{blue}{0.917}} & \cellcolor{white}{\textcolor{blue}{0.897}} & \cellcolor{white}{\textcolor{red}{0.884}} & \cellcolor{white}{\textcolor{blue}{0.896}} & \cellcolor{white}{\textcolor{red}{0.889}} & \cellcolor{white}{\textcolor{red}{0.881}} & \cellcolor{white}{\textcolor{blue}{0.894}}\\
\hline
\rowcolor{gray!6}  normal & asintotico & beta1 & \cellcolor{gray}{\textcolor{red}{0.841}} & \cellcolor{white}{\textcolor{blue}{0.891}} & \cellcolor{white}{\textcolor{blue}{0.894}} & \cellcolor{white}{\textcolor{blue}{0.897}} & \cellcolor{white}{\textcolor{red}{0.885}} & \cellcolor{white}{\textcolor{blue}{0.893}} & \cellcolor{white}{\textcolor{red}{0.872}} & \cellcolor{white}{\textcolor{red}{0.885}} & \cellcolor{white}{\textcolor{red}{0.878}}\\
\hline
normal & asintotico & beta4 & \cellcolor{gray}{\textcolor{red}{0.852}} & \cellcolor{white}{\textcolor{red}{0.888}} & \cellcolor{white}{\textcolor{blue}{0.894}} & \cellcolor{white}{\textcolor{blue}{0.91}} & \cellcolor{white}{\textcolor{blue}{0.916}} & \cellcolor{white}{\textcolor{blue}{0.897}} & \cellcolor{white}{\textcolor{blue}{0.911}} & \cellcolor{white}{\textcolor{blue}{0.907}} & \cellcolor{white}{\textcolor{blue}{0.896}}\\
\hline
\rowcolor{gray!6}  normal & exacto & beta1 & \cellcolor{white}{\textcolor{blue}{0.899}} & \cellcolor{white}{\textcolor{blue}{0.909}} & \cellcolor{white}{\textcolor{blue}{0.898}} & \cellcolor{white}{\textcolor{blue}{0.898}} & \cellcolor{white}{\textcolor{red}{0.887}} & \cellcolor{white}{\textcolor{blue}{0.893}} & \cellcolor{white}{\textcolor{red}{0.874}} & \cellcolor{white}{\textcolor{red}{0.885}} & \cellcolor{white}{\textcolor{red}{0.878}}\\
\hline
normal & exacto & beta4 & \cellcolor{white}{\textcolor{blue}{0.91}} & \cellcolor{white}{\textcolor{blue}{0.905}} & \cellcolor{white}{\textcolor{blue}{0.899}} & \cellcolor{white}{\textcolor{blue}{0.912}} & \cellcolor{white}{\textcolor{blue}{0.916}} & \cellcolor{white}{\textcolor{blue}{0.897}} & \cellcolor{white}{\textcolor{blue}{0.912}} & \cellcolor{white}{\textcolor{blue}{0.907}} & \cellcolor{white}{\textcolor{blue}{0.896}}\\
\hline
\rowcolor{gray!6}  student1 & asintotico & beta1 & \cellcolor{gray}{\textcolor{red}{0.858}} & \cellcolor{white}{\textcolor{red}{0.889}} & \cellcolor{white}{\textcolor{blue}{0.919}} & \cellcolor{white}{\textcolor{blue}{0.904}} & \cellcolor{white}{\textcolor{blue}{0.904}} & \cellcolor{white}{\textcolor{blue}{0.9}} & \cellcolor{white}{\textcolor{blue}{0.913}} & \cellcolor{white}{\textcolor{blue}{0.914}} & \cellcolor{white}{\textcolor{blue}{0.899}}\\
\hline
student1 & asintotico & beta4 & \cellcolor{gray}{\textcolor{red}{0.856}} & \cellcolor{white}{\textcolor{red}{0.878}} & \cellcolor{white}{\textcolor{blue}{0.914}} & \cellcolor{white}{\textcolor{blue}{0.906}} & \cellcolor{white}{\textcolor{blue}{0.901}} & \cellcolor{white}{\textcolor{blue}{0.908}} & \cellcolor{white}{\textcolor{blue}{0.903}} & \cellcolor{white}{\textcolor{blue}{0.908}} & \cellcolor{white}{\textcolor{blue}{0.908}}\\
\hline
\rowcolor{gray!6}  student1 & exacto & beta1 & \cellcolor{white}{\textcolor{blue}{0.923}} & \cellcolor{white}{\textcolor{blue}{0.906}} & \cellcolor{white}{\textcolor{blue}{0.921}} & \cellcolor{white}{\textcolor{blue}{0.906}} & \cellcolor{white}{\textcolor{blue}{0.905}} & \cellcolor{white}{\textcolor{blue}{0.9}} & \cellcolor{white}{\textcolor{blue}{0.914}} & \cellcolor{white}{\textcolor{blue}{0.914}} & \cellcolor{white}{\textcolor{blue}{0.899}}\\
\hline
student1 & exacto & beta4 & \cellcolor{white}{\textcolor{blue}{0.913}} & \cellcolor{white}{\textcolor{blue}{0.9}} & \cellcolor{white}{\textcolor{blue}{0.917}} & \cellcolor{white}{\textcolor{blue}{0.907}} & \cellcolor{white}{\textcolor{blue}{0.901}} & \cellcolor{white}{\textcolor{blue}{0.909}} & \cellcolor{white}{\textcolor{blue}{0.903}} & \cellcolor{white}{\textcolor{blue}{0.908}} & \cellcolor{white}{\textcolor{blue}{0.908}}\\
\hline
\rowcolor{gray!6}  student3 & asintotico & beta1 & \cellcolor{gray}{\textcolor{red}{0.853}} & \cellcolor{white}{\textcolor{red}{0.884}} & \cellcolor{white}{\textcolor{red}{0.884}} & \cellcolor{white}{\textcolor{red}{0.887}} & \cellcolor{white}{\textcolor{blue}{0.891}} & \cellcolor{white}{\textcolor{blue}{0.907}} & \cellcolor{white}{\textcolor{blue}{0.891}} & \cellcolor{white}{\textcolor{red}{0.885}} & \cellcolor{white}{\textcolor{blue}{0.894}}\\
\hline
student3 & asintotico & beta4 & \cellcolor{gray}{\textcolor{red}{0.856}} & \cellcolor{white}{\textcolor{red}{0.875}} & \cellcolor{white}{\textcolor{blue}{0.897}} & \cellcolor{white}{\textcolor{blue}{0.912}} & \cellcolor{white}{\textcolor{blue}{0.899}} & \cellcolor{white}{\textcolor{blue}{0.901}} & \cellcolor{white}{\textcolor{blue}{0.905}} & \cellcolor{white}{\textcolor{blue}{0.908}} & \cellcolor{white}{\textcolor{blue}{0.901}}\\
\hline
\rowcolor{gray!6}  student3 & exacto & beta1 & \cellcolor{white}{\textcolor{blue}{0.909}} & \cellcolor{white}{\textcolor{blue}{0.9}} & \cellcolor{white}{\textcolor{blue}{0.89}} & \cellcolor{white}{\textcolor{red}{0.889}} & \cellcolor{white}{\textcolor{blue}{0.891}} & \cellcolor{white}{\textcolor{blue}{0.908}} & \cellcolor{white}{\textcolor{blue}{0.891}} & \cellcolor{white}{\textcolor{red}{0.885}} & \cellcolor{white}{\textcolor{blue}{0.894}}\\
\hline
student3 & exacto & beta4 & \cellcolor{white}{\textcolor{blue}{0.909}} & \cellcolor{white}{\textcolor{blue}{0.898}} & \cellcolor{white}{\textcolor{blue}{0.899}} & \cellcolor{white}{\textcolor{blue}{0.912}} & \cellcolor{white}{\textcolor{blue}{0.899}} & \cellcolor{white}{\textcolor{blue}{0.901}} & \cellcolor{white}{\textcolor{blue}{0.905}} & \cellcolor{white}{\textcolor{blue}{0.91}} & \cellcolor{white}{\textcolor{blue}{0.901}}\\
\hline
\rowcolor{gray!6}  uniforme & asintotico & beta1 & \cellcolor{gray}{\textcolor{red}{0.858}} & \cellcolor{white}{\textcolor{red}{0.881}} & \cellcolor{white}{\textcolor{blue}{0.905}} & \cellcolor{white}{\textcolor{blue}{0.91}} & \cellcolor{white}{\textcolor{blue}{0.893}} & \cellcolor{white}{\textcolor{blue}{0.911}} & \cellcolor{white}{\textcolor{blue}{0.913}} & \cellcolor{white}{\textcolor{blue}{0.898}} & \cellcolor{white}{\textcolor{blue}{0.894}}\\
\hline
uniforme & asintotico & beta4 & \cellcolor{gray}{\textcolor{red}{0.823}} & \cellcolor{white}{\textcolor{blue}{0.892}} & \cellcolor{white}{\textcolor{blue}{0.895}} & \cellcolor{white}{\textcolor{blue}{0.891}} & \cellcolor{white}{\textcolor{blue}{0.897}} & \cellcolor{white}{\textcolor{blue}{0.904}} & \cellcolor{white}{\textcolor{blue}{0.91}} & \cellcolor{white}{\textcolor{blue}{0.897}} & \cellcolor{white}{\textcolor{blue}{0.891}}\\
\hline
\rowcolor{gray!6}  uniforme & exacto & beta1 & \cellcolor{white}{\textcolor{blue}{0.903}} & \cellcolor{white}{\textcolor{blue}{0.898}} & \cellcolor{white}{\textcolor{blue}{0.906}} & \cellcolor{white}{\textcolor{blue}{0.911}} & \cellcolor{white}{\textcolor{blue}{0.893}} & \cellcolor{white}{\textcolor{blue}{0.911}} & \cellcolor{white}{\textcolor{blue}{0.913}} & \cellcolor{white}{\textcolor{blue}{0.898}} & \cellcolor{white}{\textcolor{blue}{0.894}}\\
\hline
uniforme & exacto & beta4 & \cellcolor{white}{\textcolor{blue}{0.9}} & \cellcolor{white}{\textcolor{blue}{0.904}} & \cellcolor{white}{\textcolor{blue}{0.898}} & \cellcolor{white}{\textcolor{blue}{0.893}} & \cellcolor{white}{\textcolor{blue}{0.898}} & \cellcolor{white}{\textcolor{blue}{0.904}} & \cellcolor{white}{\textcolor{blue}{0.911}} & \cellcolor{white}{\textcolor{blue}{0.897}} & \cellcolor{white}{\textcolor{blue}{0.891}}\\
\hline
\end{tabular}
\end{table}

\hypertarget{para-b3000}{%
\subsubsection{\texorpdfstring{Para
\(B=3000\)}{Para B=3000}}\label{para-b3000}}

\begin{Shaded}
\begin{Highlighting}[]
\NormalTok{sintesis <-}\StringTok{ }\NormalTok{intervalos }\OperatorTok
\StringTok{  }\KeywordTok{group_by}\NormalTok{(distr_eps, n, met_int, fun_a) }\OperatorTok
\StringTok{  }\KeywordTok{summarise}\NormalTok{(}\DataTypeTok{prop_cubre =} \KeywordTok{mean}\NormalTok{(cubre))}

\NormalTok{sintesis }\OperatorTok
\StringTok{  }\KeywordTok{mutate}\NormalTok{ (}
    \DataTypeTok{prop_cubre =} \KeywordTok{round}\NormalTok{(}\DataTypeTok{digits=}\DecValTok{4}\NormalTok{,}\DataTypeTok{x=}\NormalTok{prop_cubre),}
    \DataTypeTok{prop_cubre =} \KeywordTok{cell_spec}\NormalTok{(prop_cubre,}\StringTok{"latex"}\NormalTok{,}
                           \DataTypeTok{color =} \KeywordTok{ifelse}\NormalTok{(prop_cubre }\OperatorTok{<}\StringTok{ }\FloatTok{0.89}\NormalTok{,}\StringTok{"red"}\NormalTok{,}\StringTok{"blue"}\NormalTok{),}
                           \DataTypeTok{background =} \KeywordTok{ifelse}\NormalTok{(prop_cubre }\OperatorTok{<}\StringTok{ }\FloatTok{0.86}\NormalTok{,}\StringTok{"gray"}\NormalTok{,}\StringTok{"white"}\NormalTok{)}
\NormalTok{                           )}
\NormalTok{  ) }\OperatorTok
\StringTok{  }\KeywordTok{spread}\NormalTok{(n,prop_cubre) }\OperatorTok
\StringTok{  }\KeywordTok{kable}\NormalTok{(}\DataTypeTok{format=}\StringTok{"latex"}\NormalTok{, }\DataTypeTok{escape =}\NormalTok{ F) }\OperatorTok
\StringTok{  }\KeywordTok{kable_styling}\NormalTok{(}\StringTok{"condensed"}\NormalTok{, }\StringTok{"striped"}\NormalTok{) }\OperatorTok
\StringTok{  }\KeywordTok{add_header_above}\NormalTok{(}\KeywordTok{c}\NormalTok{(}\StringTok{" "}\NormalTok{=}\DecValTok{3}\NormalTok{,}\StringTok{"valores de n"}\NormalTok{=}\DecValTok{9}\NormalTok{))}
\end{Highlighting}
\end{Shaded}

\begin{table}[H]
\centering
\begin{tabular}{l|l|l|l|l|l|l|l|l|l|l|l}
\hline
\multicolumn{3}{c|}{ } & \multicolumn{9}{c}{valores de n} \\
\cline{4-12}
distr_eps & met_int & fun_a & 10 & 25 & 100 & 250 & 500 & 1000 & 1500 & 2000 & 3000\\
\hline
\rowcolor{gray!6}  chi_cuadrado & asintotico & beta1 & \cellcolor{gray}{\textcolor{red}{0.84}} & \cellcolor{white}{\textcolor{blue}{0.8903}} & \cellcolor{white}{\textcolor{blue}{0.9013}} & \cellcolor{white}{\textcolor{blue}{0.8953}} & \cellcolor{white}{\textcolor{blue}{0.9023}} & \cellcolor{white}{\textcolor{blue}{0.8963}} & \cellcolor{white}{\textcolor{blue}{0.897}} & \cellcolor{white}{\textcolor{blue}{0.8983}} & \cellcolor{white}{\textcolor{blue}{0.9023}}\\
\hline
chi_cuadrado & asintotico & beta4 & \cellcolor{gray}{\textcolor{red}{0.8463}} & \cellcolor{white}{\textcolor{red}{0.8853}} & \cellcolor{white}{\textcolor{blue}{0.8957}} & \cellcolor{white}{\textcolor{blue}{0.898}} & \cellcolor{white}{\textcolor{blue}{0.9127}} & \cellcolor{white}{\textcolor{blue}{0.8987}} & \cellcolor{white}{\textcolor{blue}{0.9103}} & \cellcolor{white}{\textcolor{blue}{0.9107}} & \cellcolor{white}{\textcolor{blue}{0.9073}}\\
\hline
\rowcolor{gray!6}  chi_cuadrado & exacto & beta1 & \cellcolor{white}{\textcolor{blue}{0.902}} & \cellcolor{white}{\textcolor{blue}{0.9073}} & \cellcolor{white}{\textcolor{blue}{0.9027}} & \cellcolor{white}{\textcolor{blue}{0.8967}} & \cellcolor{white}{\textcolor{blue}{0.9033}} & \cellcolor{white}{\textcolor{blue}{0.8967}} & \cellcolor{white}{\textcolor{blue}{0.897}} & \cellcolor{white}{\textcolor{blue}{0.8987}} & \cellcolor{white}{\textcolor{blue}{0.9023}}\\
\hline
chi_cuadrado & exacto & beta4 & \cellcolor{white}{\textcolor{blue}{0.9057}} & \cellcolor{white}{\textcolor{blue}{0.9027}} & \cellcolor{white}{\textcolor{blue}{0.8983}} & \cellcolor{white}{\textcolor{blue}{0.899}} & \cellcolor{white}{\textcolor{blue}{0.9133}} & \cellcolor{white}{\textcolor{blue}{0.899}} & \cellcolor{white}{\textcolor{blue}{0.9103}} & \cellcolor{white}{\textcolor{blue}{0.9107}} & \cellcolor{white}{\textcolor{blue}{0.9077}}\\
\hline
\rowcolor{gray!6}  exponencial & asintotico & beta1 & \cellcolor{gray}{\textcolor{red}{0.8527}} & \cellcolor{white}{\textcolor{red}{0.8717}} & \cellcolor{white}{\textcolor{blue}{0.8957}} & \cellcolor{white}{\textcolor{blue}{0.8967}} & \cellcolor{white}{\textcolor{red}{0.8873}} & \cellcolor{white}{\textcolor{blue}{0.8943}} & \cellcolor{white}{\textcolor{blue}{0.896}} & \cellcolor{white}{\textcolor{blue}{0.8903}} & \cellcolor{white}{\textcolor{red}{0.8867}}\\
\hline
exponencial & asintotico & beta4 & \cellcolor{gray}{\textcolor{red}{0.8503}} & \cellcolor{white}{\textcolor{blue}{0.8907}} & \cellcolor{white}{\textcolor{blue}{0.901}} & \cellcolor{white}{\textcolor{blue}{0.909}} & \cellcolor{white}{\textcolor{blue}{0.905}} & \cellcolor{white}{\textcolor{blue}{0.8997}} & \cellcolor{white}{\textcolor{blue}{0.8963}} & \cellcolor{white}{\textcolor{blue}{0.9023}} & \cellcolor{white}{\textcolor{blue}{0.902}}\\
\hline
\rowcolor{gray!6}  exponencial & exacto & beta1 & \cellcolor{white}{\textcolor{blue}{0.9087}} & \cellcolor{white}{\textcolor{blue}{0.892}} & \cellcolor{white}{\textcolor{blue}{0.9003}} & \cellcolor{white}{\textcolor{blue}{0.8987}} & \cellcolor{white}{\textcolor{red}{0.8877}} & \cellcolor{white}{\textcolor{blue}{0.8953}} & \cellcolor{white}{\textcolor{blue}{0.896}} & \cellcolor{white}{\textcolor{blue}{0.891}} & \cellcolor{white}{\textcolor{red}{0.8867}}\\
\hline
exponencial & exacto & beta4 & \cellcolor{white}{\textcolor{blue}{0.907}} & \cellcolor{white}{\textcolor{blue}{0.9053}} & \cellcolor{white}{\textcolor{blue}{0.9053}} & \cellcolor{white}{\textcolor{blue}{0.9097}} & \cellcolor{white}{\textcolor{blue}{0.9063}} & \cellcolor{white}{\textcolor{blue}{0.9}} & \cellcolor{white}{\textcolor{blue}{0.8963}} & \cellcolor{white}{\textcolor{blue}{0.9027}} & \cellcolor{white}{\textcolor{blue}{0.902}}\\
\hline
\rowcolor{gray!6}  lognormal & asintotico & beta1 & \cellcolor{gray}{\textcolor{red}{0.8377}} & \cellcolor{white}{\textcolor{blue}{0.894}} & \cellcolor{white}{\textcolor{blue}{0.9}} & \cellcolor{white}{\textcolor{blue}{0.9047}} & \cellcolor{white}{\textcolor{blue}{0.8913}} & \cellcolor{white}{\textcolor{blue}{0.896}} & \cellcolor{white}{\textcolor{blue}{0.8963}} & \cellcolor{white}{\textcolor{blue}{0.901}} & \cellcolor{white}{\textcolor{blue}{0.9103}}\\
\hline
lognormal & asintotico & beta4 & \cellcolor{gray}{\textcolor{red}{0.8353}} & \cellcolor{white}{\textcolor{red}{0.8823}} & \cellcolor{white}{\textcolor{blue}{0.9113}} & \cellcolor{white}{\textcolor{blue}{0.8993}} & \cellcolor{white}{\textcolor{blue}{0.891}} & \cellcolor{white}{\textcolor{blue}{0.8913}} & \cellcolor{white}{\textcolor{blue}{0.8917}} & \cellcolor{white}{\textcolor{blue}{0.8917}} & \cellcolor{white}{\textcolor{blue}{0.9067}}\\
\hline
\rowcolor{gray!6}  lognormal & exacto & beta1 & \cellcolor{white}{\textcolor{blue}{0.897}} & \cellcolor{white}{\textcolor{blue}{0.907}} & \cellcolor{white}{\textcolor{blue}{0.9027}} & \cellcolor{white}{\textcolor{blue}{0.9053}} & \cellcolor{white}{\textcolor{blue}{0.8917}} & \cellcolor{white}{\textcolor{blue}{0.896}} & \cellcolor{white}{\textcolor{blue}{0.8963}} & \cellcolor{white}{\textcolor{blue}{0.901}} & \cellcolor{white}{\textcolor{blue}{0.9103}}\\
\hline
lognormal & exacto & beta4 & \cellcolor{white}{\textcolor{blue}{0.8987}} & \cellcolor{white}{\textcolor{blue}{0.9}} & \cellcolor{white}{\textcolor{blue}{0.9137}} & \cellcolor{white}{\textcolor{blue}{0.9007}} & \cellcolor{white}{\textcolor{blue}{0.8917}} & \cellcolor{white}{\textcolor{blue}{0.8917}} & \cellcolor{white}{\textcolor{blue}{0.8923}} & \cellcolor{white}{\textcolor{blue}{0.8917}} & \cellcolor{white}{\textcolor{blue}{0.9067}}\\
\hline
\rowcolor{gray!6}  normal & asintotico & beta1 & \cellcolor{gray}{\textcolor{red}{0.8473}} & \cellcolor{white}{\textcolor{red}{0.8883}} & \cellcolor{white}{\textcolor{blue}{0.897}} & \cellcolor{white}{\textcolor{blue}{0.896}} & \cellcolor{white}{\textcolor{red}{0.8867}} & \cellcolor{white}{\textcolor{blue}{0.896}} & \cellcolor{white}{\textcolor{blue}{0.8937}} & \cellcolor{white}{\textcolor{blue}{0.89}} & \cellcolor{white}{\textcolor{blue}{0.8917}}\\
\hline
normal & asintotico & beta4 & \cellcolor{gray}{\textcolor{red}{0.8527}} & \cellcolor{white}{\textcolor{red}{0.8883}} & \cellcolor{white}{\textcolor{blue}{0.8983}} & \cellcolor{white}{\textcolor{blue}{0.906}} & \cellcolor{white}{\textcolor{blue}{0.905}} & \cellcolor{white}{\textcolor{blue}{0.8957}} & \cellcolor{white}{\textcolor{blue}{0.9017}} & \cellcolor{white}{\textcolor{blue}{0.8973}} & \cellcolor{white}{\textcolor{blue}{0.895}}\\
\hline
\rowcolor{gray!6}  normal & exacto & beta1 & \cellcolor{white}{\textcolor{blue}{0.907}} & \cellcolor{white}{\textcolor{blue}{0.9013}} & \cellcolor{white}{\textcolor{blue}{0.901}} & \cellcolor{white}{\textcolor{blue}{0.897}} & \cellcolor{white}{\textcolor{red}{0.8873}} & \cellcolor{white}{\textcolor{blue}{0.896}} & \cellcolor{white}{\textcolor{blue}{0.8947}} & \cellcolor{white}{\textcolor{blue}{0.89}} & \cellcolor{white}{\textcolor{blue}{0.8917}}\\
\hline
normal & exacto & beta4 & \cellcolor{white}{\textcolor{blue}{0.9083}} & \cellcolor{white}{\textcolor{blue}{0.9023}} & \cellcolor{white}{\textcolor{blue}{0.902}} & \cellcolor{white}{\textcolor{blue}{0.9073}} & \cellcolor{white}{\textcolor{blue}{0.9053}} & \cellcolor{white}{\textcolor{blue}{0.8967}} & \cellcolor{white}{\textcolor{blue}{0.902}} & \cellcolor{white}{\textcolor{blue}{0.8973}} & \cellcolor{white}{\textcolor{blue}{0.895}}\\
\hline
\rowcolor{gray!6}  student1 & asintotico & beta1 & \cellcolor{gray}{\textcolor{red}{0.839}} & \cellcolor{white}{\textcolor{red}{0.8893}} & \cellcolor{white}{\textcolor{blue}{0.9033}} & \cellcolor{white}{\textcolor{blue}{0.91}} & \cellcolor{white}{\textcolor{blue}{0.906}} & \cellcolor{white}{\textcolor{blue}{0.905}} & \cellcolor{white}{\textcolor{blue}{0.9087}} & \cellcolor{white}{\textcolor{blue}{0.9103}} & \cellcolor{white}{\textcolor{blue}{0.9047}}\\
\hline
student1 & asintotico & beta4 & \cellcolor{gray}{\textcolor{red}{0.8533}} & \cellcolor{white}{\textcolor{red}{0.882}} & \cellcolor{white}{\textcolor{blue}{0.9043}} & \cellcolor{white}{\textcolor{blue}{0.913}} & \cellcolor{white}{\textcolor{blue}{0.911}} & \cellcolor{white}{\textcolor{blue}{0.9147}} & \cellcolor{white}{\textcolor{blue}{0.9057}} & \cellcolor{white}{\textcolor{blue}{0.9043}} & \cellcolor{white}{\textcolor{blue}{0.9033}}\\
\hline
\rowcolor{gray!6}  student1 & exacto & beta1 & \cellcolor{white}{\textcolor{blue}{0.911}} & \cellcolor{white}{\textcolor{blue}{0.9057}} & \cellcolor{white}{\textcolor{blue}{0.9067}} & \cellcolor{white}{\textcolor{blue}{0.9117}} & \cellcolor{white}{\textcolor{blue}{0.907}} & \cellcolor{white}{\textcolor{blue}{0.9053}} & \cellcolor{white}{\textcolor{blue}{0.909}} & \cellcolor{white}{\textcolor{blue}{0.9103}} & \cellcolor{white}{\textcolor{blue}{0.9047}}\\
\hline
student1 & exacto & beta4 & \cellcolor{white}{\textcolor{blue}{0.915}} & \cellcolor{white}{\textcolor{blue}{0.9003}} & \cellcolor{white}{\textcolor{blue}{0.9077}} & \cellcolor{white}{\textcolor{blue}{0.915}} & \cellcolor{white}{\textcolor{blue}{0.9117}} & \cellcolor{white}{\textcolor{blue}{0.915}} & \cellcolor{white}{\textcolor{blue}{0.9057}} & \cellcolor{white}{\textcolor{blue}{0.9047}} & \cellcolor{white}{\textcolor{blue}{0.9037}}\\
\hline
\rowcolor{gray!6}  student3 & asintotico & beta1 & \cellcolor{gray}{\textcolor{red}{0.8373}} & \cellcolor{white}{\textcolor{blue}{0.8913}} & \cellcolor{white}{\textcolor{blue}{0.8983}} & \cellcolor{white}{\textcolor{blue}{0.896}} & \cellcolor{white}{\textcolor{blue}{0.896}} & \cellcolor{white}{\textcolor{blue}{0.896}} & \cellcolor{white}{\textcolor{blue}{0.8953}} & \cellcolor{white}{\textcolor{blue}{0.8957}} & \cellcolor{white}{\textcolor{blue}{0.8997}}\\
\hline
student3 & asintotico & beta4 & \cellcolor{gray}{\textcolor{red}{0.8447}} & \cellcolor{white}{\textcolor{red}{0.8743}} & \cellcolor{white}{\textcolor{blue}{0.898}} & \cellcolor{white}{\textcolor{blue}{0.8957}} & \cellcolor{white}{\textcolor{blue}{0.8927}} & \cellcolor{white}{\textcolor{blue}{0.8957}} & \cellcolor{white}{\textcolor{blue}{0.898}} & \cellcolor{white}{\textcolor{blue}{0.9073}} & \cellcolor{white}{\textcolor{blue}{0.9093}}\\
\hline
\rowcolor{gray!6}  student3 & exacto & beta1 & \cellcolor{white}{\textcolor{blue}{0.8973}} & \cellcolor{white}{\textcolor{blue}{0.91}} & \cellcolor{white}{\textcolor{blue}{0.9017}} & \cellcolor{white}{\textcolor{blue}{0.898}} & \cellcolor{white}{\textcolor{blue}{0.896}} & \cellcolor{white}{\textcolor{blue}{0.8967}} & \cellcolor{white}{\textcolor{blue}{0.8953}} & \cellcolor{white}{\textcolor{blue}{0.8957}} & \cellcolor{white}{\textcolor{blue}{0.8997}}\\
\hline
student3 & exacto & beta4 & \cellcolor{white}{\textcolor{blue}{0.9}} & \cellcolor{white}{\textcolor{blue}{0.893}} & \cellcolor{white}{\textcolor{blue}{0.9013}} & \cellcolor{white}{\textcolor{blue}{0.897}} & \cellcolor{white}{\textcolor{blue}{0.8933}} & \cellcolor{white}{\textcolor{blue}{0.8957}} & \cellcolor{white}{\textcolor{blue}{0.898}} & \cellcolor{white}{\textcolor{blue}{0.908}} & \cellcolor{white}{\textcolor{blue}{0.9093}}\\
\hline
\rowcolor{gray!6}  uniforme & asintotico & beta1 & \cellcolor{gray}{\textcolor{red}{0.84}} & \cellcolor{white}{\textcolor{red}{0.885}} & \cellcolor{white}{\textcolor{blue}{0.8993}} & \cellcolor{white}{\textcolor{blue}{0.8917}} & \cellcolor{white}{\textcolor{blue}{0.895}} & \cellcolor{white}{\textcolor{blue}{0.9007}} & \cellcolor{white}{\textcolor{blue}{0.893}} & \cellcolor{white}{\textcolor{blue}{0.893}} & \cellcolor{white}{\textcolor{blue}{0.8937}}\\
\hline
uniforme & asintotico & beta4 & \cellcolor{gray}{\textcolor{red}{0.828}} & \cellcolor{white}{\textcolor{red}{0.8833}} & \cellcolor{white}{\textcolor{blue}{0.8947}} & \cellcolor{white}{\textcolor{blue}{0.8983}} & \cellcolor{white}{\textcolor{blue}{0.9013}} & \cellcolor{white}{\textcolor{blue}{0.9023}} & \cellcolor{white}{\textcolor{blue}{0.9023}} & \cellcolor{white}{\textcolor{blue}{0.8963}} & \cellcolor{white}{\textcolor{blue}{0.898}}\\
\hline
\rowcolor{gray!6}  uniforme & exacto & beta1 & \cellcolor{white}{\textcolor{blue}{0.8993}} & \cellcolor{white}{\textcolor{blue}{0.9}} & \cellcolor{white}{\textcolor{blue}{0.903}} & \cellcolor{white}{\textcolor{blue}{0.8927}} & \cellcolor{white}{\textcolor{blue}{0.896}} & \cellcolor{white}{\textcolor{blue}{0.9007}} & \cellcolor{white}{\textcolor{blue}{0.8933}} & \cellcolor{white}{\textcolor{blue}{0.893}} & \cellcolor{white}{\textcolor{blue}{0.8937}}\\
\hline
uniforme & exacto & beta4 & \cellcolor{white}{\textcolor{blue}{0.8993}} & \cellcolor{white}{\textcolor{blue}{0.896}} & \cellcolor{white}{\textcolor{blue}{0.899}} & \cellcolor{white}{\textcolor{blue}{0.9}} & \cellcolor{white}{\textcolor{blue}{0.902}} & \cellcolor{white}{\textcolor{blue}{0.9027}} & \cellcolor{white}{\textcolor{blue}{0.9027}} & \cellcolor{white}{\textcolor{blue}{0.8963}} & \cellcolor{white}{\textcolor{blue}{0.898}}\\
\hline
\end{tabular}
\end{table}

Observamos que para muestras pequeñas, con \(n=10\), es usual que los
intervalos asintóticos no lleguen a cubrir los parámetros. En varios
casos incluido el exponencial nuestra media de aciertos está por debajo
del \(\alpha\) establecido. Esto no mejora incrementando las
repeticiones hasta 3.000, es decir que probablemente ya estábamos cerca
de su límite.

\hypertarget{proporcion-de-intervalos-que-no-cubren-el-parametro-del-pgd}{%
\subsubsection{Proporción de Intervalos que no cubren el parámetro del
PGD}\label{proporcion-de-intervalos-que-no-cubren-el-parametro-del-pgd}}

En este gráfico mostramos en columnas paralelas la proporción de casos
donde no se cubrió el parámetro por método y por tamaño de muestra para
cada variable considerada.

\hypertarget{para-el-error-exponencial}{%
\paragraph{Para el Error Exponencial}\label{para-el-error-exponencial}}

\begin{Shaded}
\begin{Highlighting}[]
\NormalTok{graficar_proporciones_cobertura <-}\StringTok{ }\ControlFlowTok{function}\NormalTok{(distr_eps)\{}
\NormalTok{  intervalos }\OperatorTok
\StringTok{    }\KeywordTok{filter}\NormalTok{(distr_eps}\OperatorTok{==}\NormalTok{distr_eps,cubre}\OperatorTok{==}\OtherTok{FALSE}\NormalTok{, n_sim }\OperatorTok{<=}\StringTok{ }\DecValTok{1000}\NormalTok{) }\OperatorTok
\StringTok{    }\KeywordTok{ggplot}\NormalTok{(}\KeywordTok{aes}\NormalTok{(}\DataTypeTok{x =}\NormalTok{ met_int, }\DataTypeTok{y=}\NormalTok{..count..}\OperatorTok{/}\KeywordTok{sum}\NormalTok{(..count..) , }\DataTypeTok{fill=}\NormalTok{n, }\DataTypeTok{color =}\NormalTok{ met_int)) }\OperatorTok{+}
\StringTok{    }\KeywordTok{geom_bar}\NormalTok{() }\OperatorTok{+}
\StringTok{    }\KeywordTok{theme}\NormalTok{(}
      \DataTypeTok{axis.text.x.bottom =} \KeywordTok{element_blank}\NormalTok{()}
\NormalTok{    ) }\OperatorTok{+}
\StringTok{    }\KeywordTok{labs}\NormalTok{(}
      \DataTypeTok{x =} \KeywordTok{element_blank}\NormalTok{(),}
      \DataTypeTok{y =} \StringTok{"Proporción de fallas"}\NormalTok{,}
      \DataTypeTok{color =} \StringTok{"Método del Intervalo"}\NormalTok{,}
      \DataTypeTok{fill =} \StringTok{"n de la muestra"}
\NormalTok{    ) }\OperatorTok{+}
\StringTok{    }\KeywordTok{facet_grid}\NormalTok{(fun_a}\OperatorTok{~}\NormalTok{n)}
\NormalTok{\}}

\NormalTok{graficar_proporciones_cobertura2 <-}\StringTok{ }\ControlFlowTok{function}\NormalTok{(distr_eps)\{}
\NormalTok{  sintesis }\OperatorTok
\StringTok{    }\KeywordTok{filter}\NormalTok{(distr_eps}\OperatorTok{==}\NormalTok{distr_eps) }\OperatorTok
\StringTok{    }\KeywordTok{ggplot}\NormalTok{(}\KeywordTok{aes}\NormalTok{(}\DataTypeTok{x =}\NormalTok{ n, }\DataTypeTok{y=}\NormalTok{ prop_cubre , }\DataTypeTok{fill=}\NormalTok{n, }\DataTypeTok{color =}\NormalTok{ met_int)) }\OperatorTok{+}
\StringTok{    }\KeywordTok{geom_smooth}\NormalTok{() }\OperatorTok{+}
\StringTok{    }\KeywordTok{theme}\NormalTok{(}
      \DataTypeTok{axis.text.x.bottom =} \KeywordTok{element_blank}\NormalTok{()}
\NormalTok{    ) }\OperatorTok{+}
\StringTok{    }\KeywordTok{facet_grid}\NormalTok{(.}\OperatorTok{~}\NormalTok{met_int)}
\NormalTok{\}}

\CommentTok{#graficar_proporciones_cobertura2('exponencial')}

\KeywordTok{graficar_proporciones_cobertura}\NormalTok{(}\StringTok{'exponencial'}\NormalTok{)}
\end{Highlighting}
\end{Shaded}

\includegraphics{informeSaleaAPDF_files/figure-latex/intervalos exponenciales que cubren-1.pdf}

\hypertarget{para-el-error-lognormal}{%
\paragraph{Para el Error Lognormal}\label{para-el-error-lognormal}}

\begin{Shaded}
\begin{Highlighting}[]
\KeywordTok{graficar_proporciones_cobertura}\NormalTok{(}\StringTok{'lognormal'}\NormalTok{)}
\end{Highlighting}
\end{Shaded}

\includegraphics{informeSaleaAPDF_files/figure-latex/intervalos lognormales que cubren-1.pdf}

\hypertarget{para-el-error-uniforme}{%
\paragraph{Para el Error Uniforme}\label{para-el-error-uniforme}}

\begin{Shaded}
\begin{Highlighting}[]
\KeywordTok{graficar_proporciones_cobertura}\NormalTok{(}\StringTok{'uniforme'}\NormalTok{)}
\end{Highlighting}
\end{Shaded}

\includegraphics{informeSaleaAPDF_files/figure-latex/intervalos uniformes que cubren-1.pdf}

\hypertarget{para-el-error-chi-cuadrado}{%
\paragraph{Para el Error Chi
Cuadrado}\label{para-el-error-chi-cuadrado}}

\begin{Shaded}
\begin{Highlighting}[]
\KeywordTok{graficar_proporciones_cobertura}\NormalTok{(}\StringTok{'chi_cuadrado'}\NormalTok{)}
\end{Highlighting}
\end{Shaded}

\includegraphics{informeSaleaAPDF_files/figure-latex/intervalos chi que cubren-1.pdf}

\hypertarget{para-el-error-t}{%
\paragraph{Para el Error T}\label{para-el-error-t}}

\begin{Shaded}
\begin{Highlighting}[]
\KeywordTok{graficar_proporciones_cobertura}\NormalTok{(}\StringTok{'student3'}\NormalTok{)}
\end{Highlighting}
\end{Shaded}

\includegraphics{informeSaleaAPDF_files/figure-latex/intervalos T que cubren-1.pdf}

\hypertarget{para-el-error-cauchy}{%
\paragraph{Para el error Cauchy}\label{para-el-error-cauchy}}

\begin{Shaded}
\begin{Highlighting}[]
\KeywordTok{graficar_proporciones_cobertura}\NormalTok{(}\StringTok{'student1'}\NormalTok{)}
\end{Highlighting}
\end{Shaded}

\includegraphics{informeSaleaAPDF_files/figure-latex/intervalos cauchy que cubren-1.pdf}

\hypertarget{algunas-observaciones-sobre-estos-patrones}{%
\subsubsection{Algunas Observaciones Sobre Estos
Patrones}\label{algunas-observaciones-sobre-estos-patrones}}

Se ven varios fenómenos que aprecen dignos de interés pero no querríamos
interpretar de forma trivial.

Los incrementos tanto en el tamaño de muestra como en el número de
repeticiones brindan en principio un incremento en el apego al valor de
\(\alpha\) para los intervalos asintóticos. Después de una cantidad de
valores vemos que ambos se comportan en forma prácticamente
indistinguible aunque es notable la precisión inicial del intervalo
exacto.

Es imposible incrementar nuestro apego en la tasa de acierto al alfa más
allá de cierto nivel, de hecho en algunos casos observamos regresiones
al incrementar el valor de \(B\). Estas regresiones son pequeñas, en la
mayoría de los casos una vez que nos acercamos lo suficiente en la
proporción de aciertos a \(1-\alpha\) permanecemos ahí, pero en el
gráfico se ven exageradas por la escala. Originalmente habíamos hecho un
gráfico de tasa de aciertos en vez de tasa de errores y eran bastante
difíciles de percibir.

En el caso de la distribución de Cauchy es tentador pensar que si bien
no hay esperanza finita, hay una región que concentra la máxima densidad
y por tanto define un intervalo de máxima probabilidad. La ausencia de
esperanza es porque las colas no convergen a 0 con suficiente
``rapidez'' y eso nos va a afectar a la hora de estudiar cómo se portan
los estimadores cuando el error tiene esta forma. El intervalo, en
cambio, funciona razonablemente bien.

\hypertarget{los-estimadores-tienen-distribucion-normal}{%
\subsection{¿Los Estimadores Tienen Distribución
Normal?}\label{los-estimadores-tienen-distribucion-normal}}

\hypertarget{normalidad-de-hatbeta_4-cuando-el-error-es-exponencial.}{%
\subsubsection{\texorpdfstring{Normalidad de \(\hat{\beta}_4\) cuando el
error es
exponencial.}{Normalidad de \textbackslash hat\{\textbackslash beta\}\_4 cuando el error es exponencial.}}\label{normalidad-de-hatbeta_4-cuando-el-error-es-exponencial.}}

El siguiente gráfico nos muestra el gráfico QQ para el parámetro
\(\hat{\beta}_4\) y su evolución al incrementar la cantidad de muestras
\(n\).

El mismo tiene una escala fija para que tenga sentido la comparación
gráfica.

\begin{Shaded}
\begin{Highlighting}[]
\NormalTok{intervalos }\OperatorTok
\StringTok{  }\KeywordTok{filter}\NormalTok{(n_sim }\OperatorTok{<=}\StringTok{ }\DecValTok{50}\NormalTok{, }\KeywordTok{is.element}\NormalTok{(n,}\KeywordTok{c}\NormalTok{(}\DecValTok{10}\NormalTok{,}\DecValTok{100}\NormalTok{,}\DecValTok{1000}\NormalTok{,}\DecValTok{3000}\NormalTok{)),fun_a}\OperatorTok{==}\StringTok{'beta4'}\NormalTok{,distr_eps }\OperatorTok{==}\StringTok{ 'exponencial'}\NormalTok{)  }\OperatorTok
\StringTok{  }\KeywordTok{ggplot}\NormalTok{() }\OperatorTok{+}
\StringTok{  }\KeywordTok{aes}\NormalTok{(}\DataTypeTok{sample =}\NormalTok{estimador, }\DataTypeTok{color =}\NormalTok{ distr_eps) }\OperatorTok{+}
\StringTok{  }\KeywordTok{geom_qq}\NormalTok{() }\OperatorTok{+}
\StringTok{  }\KeywordTok{stat_qq_line}\NormalTok{() }\OperatorTok{+}
\StringTok{  }\KeywordTok{facet_grid}\NormalTok{(n}\OperatorTok{~}\NormalTok{.) }\OperatorTok{+}
\StringTok{  }\KeywordTok{labs}\NormalTok{(}
    \DataTypeTok{x =} \StringTok{"Muestra"}\NormalTok{,}
    \DataTypeTok{y =} \StringTok{"Teórico"}
\StringTok{  ) +}
\StringTok{  guides( color = FALSE )}
\end{Highlighting}
\end{Shaded}

\includegraphics{informeSaleaAPDF_files/figure-latex/qq exponencial-1.pdf}

Es interesante observar qué pasa si vamos reduciendo la escala también:
El patrón de distancia a la recta de los cuantiles ideales se mantiene
similar, aunque estas se ven reducidas hasta tal punto que en el otro
gráfico donde mantenemos la escala constante parecen adherirse a ella.
Pareciera observarse el efecto aritmético de dividir por \(n\),
manteniéndose la relación de aspecto entre las distancias.

\begin{Shaded}
\begin{Highlighting}[]
\NormalTok{intervalos }\OperatorTok
\StringTok{  }\KeywordTok{filter}\NormalTok{(n_sim }\OperatorTok{<=}\StringTok{ }\DecValTok{50}\NormalTok{, }\KeywordTok{is.element}\NormalTok{(n,}\KeywordTok{c}\NormalTok{(}\DecValTok{10}\NormalTok{,}\DecValTok{100}\NormalTok{,}\DecValTok{1000}\NormalTok{,}\DecValTok{3000}\NormalTok{)),fun_a}\OperatorTok{==}\StringTok{'beta4'}\NormalTok{,distr_eps }\OperatorTok{==}\StringTok{ 'exponencial'}\NormalTok{)  }\OperatorTok
\StringTok{  }\KeywordTok{ggplot}\NormalTok{() }\OperatorTok{+}
\StringTok{  }\KeywordTok{aes}\NormalTok{(}\DataTypeTok{sample =}\NormalTok{estimador, }\DataTypeTok{color =}\NormalTok{ distr_eps) }\OperatorTok{+}
\StringTok{  }\KeywordTok{geom_qq}\NormalTok{() }\OperatorTok{+}
\StringTok{  }\KeywordTok{stat_qq_line}\NormalTok{() }\OperatorTok{+}
\StringTok{  }\KeywordTok{facet_grid}\NormalTok{(n}\OperatorTok{~}\NormalTok{., }\DataTypeTok{scales=}\StringTok{"free"}\NormalTok{) }\OperatorTok{+}
\StringTok{  }\KeywordTok{labs}\NormalTok{(}
    \DataTypeTok{x =} \StringTok{"Muestra"}\NormalTok{,}
    \DataTypeTok{y =} \StringTok{"Teórico"}
\StringTok{  ) +}
\StringTok{  guides( color = FALSE )}
\end{Highlighting}
\end{Shaded}

\includegraphics{informeSaleaAPDF_files/figure-latex/qq exponencial libre-1.pdf}

\hypertarget{normalidad-de-hatbeta_4-para-los-diversos-errores-segun-n.}{%
\subsubsection{\texorpdfstring{Normalidad de \(\hat{\beta}_4\) para los
diversos errores según
\(n\).}{Normalidad de \textbackslash hat\{\textbackslash beta\}\_4 para los diversos errores según n.}}\label{normalidad-de-hatbeta_4-para-los-diversos-errores-segun-n.}}

\begin{Shaded}
\begin{Highlighting}[]
\NormalTok{intervalos }\OperatorTok
\StringTok{  }\KeywordTok{filter}\NormalTok{(n_sim }\OperatorTok{<=}\StringTok{ }\DecValTok{50}\NormalTok{, }\KeywordTok{is.element}\NormalTok{(n,}\KeywordTok{c}\NormalTok{(}\DecValTok{10}\NormalTok{,}\DecValTok{100}\NormalTok{,}\DecValTok{1000}\NormalTok{,}\DecValTok{3000}\NormalTok{)),fun_a}\OperatorTok{==}\StringTok{'beta1'}\NormalTok{)  }\OperatorTok
\StringTok{  }\KeywordTok{ggplot}\NormalTok{() }\OperatorTok{+}
\StringTok{  }\KeywordTok{aes}\NormalTok{(}\DataTypeTok{sample =}\NormalTok{estimador, }\DataTypeTok{color =}\NormalTok{ distr_eps) }\OperatorTok{+}
\StringTok{  }\KeywordTok{geom_qq}\NormalTok{() }\OperatorTok{+}
\StringTok{  }\KeywordTok{stat_qq_line}\NormalTok{() }\OperatorTok{+}
\StringTok{  }\KeywordTok{facet_grid}\NormalTok{(n}\OperatorTok{~}\NormalTok{distr_eps) }\OperatorTok{+}
\StringTok{  }\KeywordTok{labs}\NormalTok{(}
    \DataTypeTok{x =} \StringTok{"Muestra"}\NormalTok{,}
    \DataTypeTok{y =} \StringTok{"Teórico"}
\StringTok{  ) +}
\StringTok{  guides( color = FALSE )}
\end{Highlighting}
\end{Shaded}

\includegraphics{informeSaleaAPDF_files/figure-latex/qqses por n y distr_eps-1.pdf}

\hypertarget{conclusiones-sobre-la-normalidad-del-estimador}{%
\paragraph{Conclusiones sobre la Normalidad del
Estimador}\label{conclusiones-sobre-la-normalidad-del-estimador}}

Como vemos, incrementando lo suficiente el valor de \(n\) esta se
termina manifestando rápidamente, aún cuando para el gráfico dejamos
fijo el valor \(B=50\), uno muy chico respecto a los totales que
manejamos, de hasta 3.000. Por supuesto, este fenómeno no se presenta
sólo en las distribuciones que cumplen las hipótesis del Teorema del
Límite central. En ese sentido la inclusión de error con distribución
\(t_1\) nos permitió ver qué pasa cuando no es correcta la aproximación
normal. En todos los casos observamos puntos que están drásticamente
lejos de los cuantiles de la distribución normal.

\hypertarget{conclusiones}{%
\section{Conclusiones}\label{conclusiones}}

En general se observan varias tendencias.

\begin{itemize}
\tightlist
\item
  Para todas las distribuciones obtuvimos mejores resultados con los
  intervalos exactos.
\item
  En casi todas al incrementar \(n\) logramos llevar el nivel de los
  intervalos asintóticos prácticamente al de los exactos, a partir de
  ciertos tamaños se mueven a la par, es decir que estiman casi
  idénticamente.
\item
  La única distribución del error que hizo que el estimador se comporte
  fuera de lo esperable fue la Cauchy, todas las pedidas por la consigna
  terminan por converger a un comportamiento asintóticamente normal.
\item
  Sería interesante buscar conjuntos de datos que no vengan de
  distribuciones teóricas para ver si en ese caso los intervalos
  asintóticos se comportan mejor.
\item
  Sería interesante hacer un estudio que contemple también la anchura de
  los intervalos.
\end{itemize}


\end{document}
